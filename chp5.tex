\chapter{总结}
在本文章中,我们主要的研究对象是基于原模图构造的有限长LDPC卷积码以及LDPC分组码。
为了公平地比较了两者的性能,本文在第二章通过类似的构造方式,即基于原模图复制和连接,构造了LDPC分组码和LDPC卷积码。
本文在第三章简单描述了用于解码LDPC分组码的和积算法,并介绍了用于解码LDPC卷积码的滑动窗口算法。
接着,我们根据置信传播算法的软判决方法,为本文的实证分析设计了用于滑动窗口解码器的迭代停止规则。

本文的实证分析表明,在LDPC卷积码的限制长度与LDPC分组码的码长相等时,(2,4)正则以及(3,6)正则q元LDPC卷积码相比于对应的LDPC分组码获得了某种“卷积优势”。
在固定解码延时的条件下,我们还比较了q元LDPC卷积码与LDPC分组码的原模图复制数、解码窗口大小以及误比特率等参数之间的关系。
在该条件下,实验结果表明,(3,6)正则非二元LDPC卷积码的性能远远优于q元LDPC卷积码以及二元LDPC卷积码的性能。
除此以外,在固定GF($q$)以及解码延时的条件下,(3,6)正则q元LDPC卷积码的解码性能随着解码窗口大小达到$W=6$而达到峰值,然后解码性能随着解码窗口继续增大而渐渐下降。
这种实验现象在(2,4)正则q元LDPC卷积码上也有体现。

在文章的最后,我们还分别计算并分析了q元LDPC卷积码与q元LDPC分组码的解码计算复杂度。
接着,本文在相等解码延时的条件下,比较了两者的解码计算复杂度。
实验结果表明,(3,6)正则4元LDPC卷积码具有较小的解码延时,较低的解码计算复杂度,以及较好的解码性能等优点。

本文的实证分析将为以后LDPC卷积码进一步应用提供参考依据。


















