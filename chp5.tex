\chapter{总结}
本文在GF($q$)上,基于LDPC分组码以及基于原模图构造的有限长LDPC卷积码展开讨论,并比较了两者的性能。

为了公平地比较了两者的性能,本文在第二章通过类似的构造方式,即基于原模图复制和连接,构造了LDPC分组码和LDPC卷积码,并保证两者的解码延时相等。
本文在第三章简单描述了用于解码LDPC分组码的和积算法,并介绍了用于解码LDPC卷积码的滑动窗口算法。

通过实现以上算法进行解码实验,模拟结果表明LDPC卷积码所隐含的卷积特性使其性能超过LDPC分组码。
另外,本文还比较了,原模图复制数、解码窗口大小以及误比特率等参数之间的联系。
从中,我们可以得到多元LDPC卷积码性能比二元LDPC卷积码、多元LDPC分组码的性能更好。
在解码延时确定的条件下,原模图复制数、解码窗口大小等参数的组合导致不同的误比特率,故应根据信道及校验矩阵等具体情况确定最优组合。本文将为以后的应用提供参考数据。