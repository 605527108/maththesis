% -*- coding: utf-8 -*-
%%
%%
%%
%%
%%
%%
%%  本模板可以使用以下两种方式编译:
%%
%%     1. PDFLaTeX
%%
%%     2. XeLaTeX [推荐]
%%
%%  注意:
%%    1. 在改变编译方式前应先删除 *.toc 和 *.aux 文件,
%%       因为不同编译方式产生的辅助文件格式可能并不相同。
%%
%%
\documentclass[12pt,openright]{book}

\usepackage{ifxetex}
\ifxetex
  \usepackage[bookmarksnumbered]{hyperref}
\else
  \usepackage[unicode,bookmarksnumbered]{hyperref}
\fi

\usepackage[emptydoublepage]{NKThesis}   % 中文
%\usepackage[emptydoublepage,English]{NKThesis} % 英文

%   根据需要选择 biblatex 宏包选项.
\usepackage[
  backend = biber, style = caspervector, utf8,
  giveninits = true, sortgiveninits = true
]{biblatex}
\hypersetup{colorlinks=true,
            pdfborder=0 0 1,
            citecolor=black,
            linkcolor=black}
\usepackage{tikz}

\addbibresource{nkthesis.bib}
\DeclareBibliographyCategory{cited}
\AtEveryCitekey{\addtocategory{cited}{\thefield{entrykey}}}

\includeonly{
abstract,
manual,
tikz,
acknowledgements,
references,
appendices,
resume,
chp1,
chp2,
chp3,
chp4,
chp5
}
\newtheorem{Theorem}{\hskip 2em 定理}[chapter]
\newtheorem{Lemma}[Theorem]{\hskip 2em 引理}
\newtheorem{Corollary}[Theorem]{\hskip 2em 推论}
\newtheorem{Proposition}[Theorem]{\hskip 2em 命题}
\newtheorem{Definition}[Theorem]{\hskip 2em 定义}
\newtheorem{Example}[Theorem]{\hskip 2em 例}
\begin{document}

%  设置基本信息
%  注意:  逗号`,'是项目分隔符. 如果某一项的值出现逗号, 应放在花括号内, 如 {,}
%
\NKTsetup{%
  论文题目(中文) = GF(q)上LDPC分组码和卷积码的比较,
  副标题         = ,
  论文题目(英文) = Comparison of LDPC Block and Convolutional Codes over GF(q),
  论文作者       = ,
  学号           = ,
  指导教师       = ,
  申请学位       = ,
  培养单位       = ,
  学科专业       = ,
  研究方向       = ,
  中图分类号     = ,
  UDC            = ,
  学校代码       = 10055,
  密级           = 公开,
                   % 公开 | 限制 | 秘密 | 机密, 若为公开, 不填以下三项
  保密期限       = ,
  审批表编号     = ,
  批准日期       = ,
  论文完成时间   = 二〇一六年五月,
  答辩日期       = ,
  论文类别       = 学士,
                   % 博士 | 学历硕士 | 硕士专业学位 | 高校教师 | 同等学力硕士
  院/系/所       = ,
  专业           = ,
  联系电话       = ,
  Email          = ,
  通讯地址(邮编) = ,
  备注           = }


% -*- coding: utf-8 -*-


\begin{zhaiyao}
本文介绍了GF(q)上LDPC分组码,并从LDPC分组码构建结构相似的卷积码。
接着分析了LDPC分组码的置信传播算法的性能,以及用于LDPC卷积码的滑动窗口版本置信传播算法的性能。
在不同参数的GF(q)上,本文考虑了两种情况,一种情况是LDPC分组码块长度与LDPC卷积码约束长度相等,另一种情况是解码延时相等,比较LDPC分组码与LDPC卷积码的解码性能。
其中模拟的结果表明q元LDPC卷积码的解码性能好与2元LDPC卷积码。
同时q元LDPC卷积码的解码性能好与q元LDPC分组码。
\end{zhaiyao}


\begin{guanjianci}
LDPC分组码;LDPC卷积码;滑动窗口;置信传播算法;解码延时
\end{guanjianci}



\begin{abstract}
In this paper, we introduce LDPC block codes (LDPC-BC), and LDPC convolutional codes (LDPC-CC) which are derived form the former. Then we analyse belief propagation (BP) decoder for LDPC-BC, and a sliding window decoder for LDPC-CC. Base on different GF(q), we compare the decoding performance between q-ary LDPC-BC and LDPC-CC in two aspects, one is the block length of q-ary LDPC-BC equals to constraint length of LDPC-CC, another is when two decoding latencies equal. Simulation shows that q-ary LDPC-CC outperform binary LDPC-CC and q-ary LDPC-BC.
\end{abstract}


\begin{keywords}
LDPC block codes; LDPC convolutional codes; sliding window decoder; belief propagation; decoding latencie.
\end{keywords} 
\tableofcontents
\chapter{背景及介绍}
Gallager在1962年提出低密度校验分组码(引文),同时他还提出了一种信息遍历解码算法。
在(引文)中,模拟结果显示,用信息遍历算法解码LDPC分组码的性能超过了turbo码,并接近香农极限。
在Low-density parity-check code over GF(q)1998,Davey和MacKay考虑了定义在GF(q)的LDPC分组码并且把Gallager的BP解码算法推广到q元,称作和积算法,并得到很好的性能。

LDPC卷积码是由Felström和Zigangirov在(引文)提出来的。
我们可以通过将几个正则LDPC码叠加连接在一起形成LDPC卷积码。
LDPC卷积码的特点是它的BP解码阈值能达到它所基于的LDPC码的最大后验概率(MAP)解码阈值(引文)。
为了解决LDPC卷积码顺序解码时占用大量内存和解码延时过长的缺点,在(引文)Windowed erasure decoding of LDPC convolutional codes中考虑了一种窗口解码,并研究了解码延时和解码性能的取舍。

由于LDPC卷积码与LDPC分组码具有相似的结构,并且在解码延时和解码性能中的取舍,一些文章已经开始研究其二者的性能比较Comparison of LDPC Block and LDPC Convolutional Codes Based on their Decoding Latency,LDPC Codes and Convolutional Codes with Equal Structural Delay: A Comparison。
本文基于以上进行q元域的推广,同时比较别的实用参数。
本文的结构如下:第一章主要是LDPC卷积码与LDPC分组码背景及介绍。第二章主要是描述LDPC卷积码与LDPC分组码的构建过程,要在特殊的构建原则上两者才能比较。第三章主要描述LDPC卷积码与LDPC分组码的解码算法,并且计算了两种算法的解码延时。第四章通过固定几个参数比较另外的参数,比较LDPC卷积码与LDPC分组码的性能。第五章是结论及总结。
\chapter{基于原模图构造的LDPC码}
\section{LDPC码的原模图及LDPC码的叠加方式}
许多好的LDPC码都是通过人工叠加短LDPC码而形成的。
LDPC码的叠加能很形象地通过Tanner图的方式表现出来。
考虑一个短LDPC码,以及与其对应的小的Tanner图,如图\ref{fig:basegraph}。
这个小的Tanner图有多个名字,如原模图、基本图或者投影图。
从原模图、基本图这两个名字可以看出,这种小的Tanner图是组成更大的Tanner图的基础。
而投影图这个名字则说明了,这个小的Tanner图是叠加其自身而构成更大的Tanner图的投影。
图\ref{fig:basegraph}中圈中等号代表变量节点,圈中加号代表校验节点,该图右侧为对应LDPC码的校验矩阵。
\begin{center}
\def\check{%
    \filldraw [fill=white,very thick] (0,0) circle (5pt);
    \draw [very thick] (0,3.5pt)--(0,-3.5pt);
    \draw [very thick] (3.5pt,0)--(-3.5pt,0);
}
\def\bit{%
    \filldraw [fill=white,very thick] (0,0) circle (5pt);
    \draw [very thick] (-3.2pt,2.2pt)--(3.2pt,2.2pt);
    \draw [very thick] (-3.2pt,-2.2pt)--(3.2pt,-2.2pt);
}
\begin{tikzpicture}
\draw [very thick] (-3,2.5) -- (-2,0);
\draw [very thick] (3,2.5) -- (-2,0);
\draw [very thick] (-1,2.5) -- (0,0);
\draw [very thick] (1,2.5) -- (0,0);
\draw [very thick] (-3,2.5) -- (2,0);
\draw [very thick] (-1,2.5) -- (2,0);
\draw [very thick] (1,2.5) -- (2,0);
\draw [very thick] (3,2.5) -- (2,0);
\begin{scope}[xshift=1cm,yshift=2.5cm]
\bit
\end{scope}
\begin{scope}[xshift=-1cm,yshift=2.5cm]
\bit
\end{scope}
\begin{scope}[xshift=3cm,yshift=2.5cm]
\bit
\end{scope}
\begin{scope}[xshift=-3cm,yshift=2.5cm]
\bit
\end{scope}
\begin{scope}[xshift=2cm,yshift=0cm]
\check
\end{scope}
\begin{scope}[xshift=-2cm,yshift=0cm]
\check
\end{scope}
\check
\draw [<->,thick] (3.3,1.25) -- (4.3,1.25);
\node at (7,1.25) (steptwo){
	\begin{minipage}{0.40\textwidth}
		\begin{equation*}
    H_b = \left(
      \begin{array}{cccc}
        1 & 0 & 0 & 1 \\
        0 & 1 & 1 & 0 \\
        1 & 1 & 1 & 1 
      \end{array} \right)
\end{equation*}
	\end{minipage}
};
\end{tikzpicture}
\figurecaption{基本图}
\label{fig:basegraph}
\end{center}

对于规则的LDPC码,可以采取更简单的原模图来描述。由于规则的LDPC码的每一个信息节点与校验节点都有相同的度分布,即相同的边连接模式,所以我们可以将基本图折叠起来,让它变成如图\ref{fig:protograph}的形状。图\ref{fig:protograph}表示的是(3,6)-正则LDPC码的原模图以及其对应的基本矩阵描述。
\begin{center}
\def\linkdoub{\draw [double distance=1mm, very thick] (0,0)--}
\def\linksing{\draw [very thick] (0,0)--}
\def\check{%
    \filldraw [fill=white,very thick] (0,0) circle (5pt);
    \draw [very thick] (0,3.5pt)--(0,-3.5pt);
    \draw [very thick] (3.5pt,0)--(-3.5pt,0);
}
\def\bit{%
    \filldraw [fill=white,very thick] (0,0) circle (5pt);
    \draw [very thick] (-3.2pt,2.2pt)--(3.2pt,2.2pt);
    \draw [very thick] (-3.2pt,-2.2pt)--(3.2pt,-2.2pt);
}
\def\thetaone{90}
\def\thetatwo{-90}
\def\armLength{0.9}
\def\symbolDist{1}
\begin{tikzpicture}
\linkdoub(\thetaone:\armLength);
\linksing(\thetaone:\armLength);
\linksing(\thetatwo:\armLength);
\begin{scope}[shift=(\thetaone:\symbolDist)]
\bit
\end{scope}
\begin{scope}[shift=(\thetatwo:\symbolDist)]
\linkdoub(\thetaone:\armLength);
\linksing(\thetaone:\armLength);
\bit
\end{scope}
\check
\draw [<->,thick] (0.5,0) -- (1.3,0);
\node [align=center] at (2.5,0) {$ \mathbf{B} = [\begin{array}{cc} 3&3 \end{array}]$};

\end{tikzpicture}
\figurecaption{LDPC码的原模图}
\label{fig:protograph}
\end{center}

为了构造有结构的LDPC码,首先要将图\ref{fig:basegraph}复制m次。
此时变量节点、校验节点以及节点间相连的边都变成了m份,形成了一个簇状的Tanner图,如图\ref{fig:copy}。
\begin{center}
\def\check{%
    \filldraw [fill=white,very thick] (0,0) circle (5pt);
    \draw [very thick] (0,3.5pt)--(0,-3.5pt);
    \draw [very thick] (3.5pt,0)--(-3.5pt,0);
}
\def\bit{%
    \filldraw [fill=white,very thick] (0,0) circle (5pt);
    \draw [very thick] (-3.2pt,2.2pt)--(3.2pt,2.2pt);
    \draw [very thick] (-3.2pt,-2.2pt)--(3.2pt,-2.2pt);
}
\begin{tikzpicture}
\foreach \x / \z in {0/0,12/3,24/6}
{
\begin{scope}[xshift=\x,yshift=\z]
	\draw[very thick] (-3,4) -- (-2,0);
	\draw[very thick] (3,4) -- (-2,0);
	\draw[very thick] (-1.2,4) -- (0,0);
	\draw[very thick] (1.2,4) -- (0,0);
	\draw[very thick] (-3,4) -- (2,0);
	\draw[very thick] (-1.2,4) -- (2,0);
	\draw[very thick] (1.2,4) -- (2,0);
	\draw[very thick] (3,4) -- (2,0);
\begin{scope}[xshift=1.2cm,yshift=4cm]
\bit
\end{scope}
\begin{scope}[xshift=-1.2cm,yshift=4cm]
\bit
\end{scope}
\begin{scope}[xshift=3cm,yshift=4cm]
\bit
\end{scope}
\begin{scope}[xshift=-3cm,yshift=4cm]
\bit
\end{scope}
\begin{scope}[xshift=2cm,yshift=0cm]
\check
\end{scope}
\begin{scope}[xshift=-2cm,yshift=0cm]
\check
\end{scope}
\check
\end{scope}
}
\end{tikzpicture}
\figurecaption{复制m次基本图}
\label{fig:copy}
\end{center}
原模图被复制了m次之后,虽然变量节点、校验节点以及节点间相连的边都变成了m份,
但是此时这m个原模图还是隔离的。
所以要对每一个边的簇应用置换方法,使得不同的原模图之间相连起来。
此时便完成了从原模图构造如原模图m倍大小的有结构的LDPC码,如图\ref{fig:structed}。

如果采用矩阵的表述方式,一开始我们有一个LDPC短码的校验矩阵$H_b$,如图\ref{fig:basegraph}的右边的矩阵所示。
令$\mathcal{P}$为$m\times m$维置换矩阵的集合。
我们通过把$H_b$中的每个元素换成$m\times m$的矩阵来形成$m$倍大的LDPC码。
其中,把$H_b$中为$1$的元素换成$\mathcal{P}$中的某个矩阵,把$H_b$中为$0$的元素换成零矩阵。
通过这种复制方式,LDPC短码的校验矩阵$H_b$变成了$m$倍大的校验矩阵:
\begin{equation}
    H = \left(
      \begin{array}{cccc}
        P^2 & 0 & 0 & P^1 \\
        0 & P^3 & P^1 & 0 \\
        P^1 & P^3 & P^2 & P^2 
      \end{array} \right)
\end{equation}
其中$P^i$对应循环置换矩阵,其意义是对集合$\{1,\dots,m\}$进行移位操作$i-1$次:
它的列向量对应于变量节点,行向量对应于校验节点。比如$P^1$对应单位矩阵,以及
\begin{equation}
    P^2 = \left(
      \begin{array}{ccc}
        0 & 0 & 1 \\
        1 & 0 & 0 \\
        0 & 1 & 0
      \end{array} \right)
\end{equation}
$P^2$表示的意思是将第一个原模图的第一个变量节点连接到第二个原模图的第一个校验节点,将第二个原模图的第一个变量节点连接到第三个原模图的第一个校验节点,将第三个原模图的第一个变量节点连接到第一个原模图的第一个校验节点。
\begin{center}
\def\check{%
    \filldraw [fill=white,very thick] (0,0) circle (5pt);
    \draw [very thick] (0,3.5pt)--(0,-3.5pt);
    \draw [very thick] (3.5pt,0)--(-3.5pt,0);
}
\def\bit{%
    \filldraw [fill=white,very thick] (0,0) circle (5pt);
    \draw [very thick] (-3.2pt,2.2pt)--(3.2pt,2.2pt);
    \draw [very thick] (-3.2pt,-2.2pt)--(3.2pt,-2.2pt);
}
\begin{tikzpicture}
%p2
  \draw[very thick] (-2.2,4.2) -- (-2,0);
  \draw[very thick] (-3,4) -- (-1.6,0.1);
  \draw[very thick] (-2.6,4.1) -- (-1.2,0.2);
%p3
  \draw[very thick] (-0.8,4.1) -- (0,0);
  \draw[very thick] (-0.4,4.2) -- (0.4,0.1);
  \draw[very thick] (-1.2,4) -- (0.8,0.2);
%p3
  \draw[very thick] (-0.8,4.1) -- (2,0);
  \draw[very thick] (-0.4,4.2) -- (2.4,0.1);
  \draw[very thick] (-1.2,4) -- (2.8,0.2);
%p2
  \draw[very thick] (2.0,4.2) -- (2,0);
  \draw[very thick] (1.2,4) -- (2.4,0.1);
  \draw[very thick] (1.6,4.1) -- (2.8,0.2);
%p2
  \draw[very thick] (3.8,4.2) -- (2,0);
  \draw[very thick] (3,4) -- (2.4,0.1);
  \draw[very thick] (3.4,4.1) -- (2.8,0.2);
\foreach \x / \z in {0/0,0.4/0.1,0.8/0.2}
{
\begin{scope}[xshift=\x cm,yshift=\z cm]
\draw[very thick] (3,4) -- (-2,0);
\draw[very thick] (1.2,4) -- (0,0);
\draw[very thick] (-3,4) -- (2,0);
\begin{scope}[xshift=1.2cm,yshift=4cm]
\bit
\end{scope}
\begin{scope}[xshift=-1.2cm,yshift=4cm]
\bit
\end{scope}
\begin{scope}[xshift=3cm,yshift=4cm]
\bit
\end{scope}
\begin{scope}[xshift=-3cm,yshift=4cm]
\bit
\end{scope}
\begin{scope}[xshift=2cm,yshift=0cm]
\check
\end{scope}
\begin{scope}[xshift=-2cm,yshift=0cm]
\check
\end{scope}
\check
\end{scope}
}
\end{tikzpicture}
\figurecaption{具有一定结构的LDPC码}
\label{fig:structed}
\end{center}

\section{LDPC分组码的构造}
设计码率为$R=b/c$的LDPC分组码的原模图有$c$个变量节点和$c-b$个校验节点。
它能生成不同码长的,设计码率为$R$,有相同的度分布的分组码。
一个有$c=2$个变量节点,变量节点的度为3,和$c-b=1$个校验节点,校验节点的度为6的分组码原模图的例子可以参看图\ref{fig:basegraph}。

下面将描述$q$元LDPC分组码的构造方法。
GF($q$)为含$q=2^m$个元素的有限域,其中$m$为在GF($q$)代表一个符号所需的位数。
令$M$为原模图叠加数,其中$M$是一个大整数。
通过以下两步从原模图的$(c-b)\times c$邻接矩阵$\mathbf{B}=[B_{i,j}]$构造码长为$n_{BC}=Mc$的$q$元LDPC分组码:
\begin{enumerate}
\item 将$\mathbf{B}$中的非零元$B_{i,j}$替换为随机选择的$M \times M$置换矩阵,将$\mathbf{B}$中的零元$B_{i,j}$替换为$M \times M$零矩阵。此时得到对应于$\mathbf{B}$的二元校验矩阵$\mathbf{H}$;
\item 将$\mathbf{H}$中非零元替换为从有限域GF($q$)中随机选取的元素,得到LDPC分组码的$q$元校验矩阵$\mathbf{H}_{BC}$。
\end{enumerate}
对于LDPC分组码,必须等待整个码块接受完毕才能执行置信传播解码算法。故$q$元LDPC分组码的解码延时为
\begin{equation}
T_{BC}=n_{BC}\cdot m = Mmc
\end{equation}

\section{LDPC卷积码的构造}

对于给定码率为$R=b/c$的LDPC卷积码,其定义为,存在校验矩阵$\mathbf{H}_{[\infty]}$使得无限长向量$\mathbf{v}_{[\infty]}$有$\mathbf{H}_{[\infty]}\mathbf{v}_{[\infty]}^\text{T} =\mathbf{0}_{[\infty]}$,其中
\begin{equation}
    \mathbf{H}_{[\infty]} = \left[
          \begin{array}{ccccc}
            \mathbf{H}_0(1) & & & & \\
            \mathbf{H}_1(1) & \mathbf{H}_0(2) & & & \\
            \vdots & \mathbf{H}_1(2) & \ddots & & \\
            \mathbf{H}_{m_s}(1) & \vdots & \ddots & \mathbf{H}_0(t) & \\
             & \mathbf{H}_{m_s}(2) & \ddots & \mathbf{H}_1(t) & \ddots\\
             & & \ddots & \vdots & \ddots \\
             & & & \mathbf{H}_{m_s}(t) & \ddots \\
             & & & & \ddots
          \end{array} \right]
\end{equation}
而$\mathbf{0}_{[\infty]}$是无限长零向量。
$\mathbf{H}_i(t),i=0,1,\dots,m_s$为$(c-b)\times c$的矩阵满足以下条件:
\begin{itemize}
\item $\mathbf{H}_i(t)=\mathbf{0}$,当$i<0$和$i>m_s$,$\forall t \geq 1$;
\item $\exists t\geq 0$使得$\mathbf{H}_{m_s}(t) \neq \mathbf{0}$;
\item $\forall t \geq 1,\mathbf{H}_0(1)$满秩。
\end{itemize}

其中参数$m_s$称为LDPC卷积码的记忆因子。
$\nu = (m_s+1)c$为LDPC卷积码的限制长度。
对于$\forall t,\tau>1,\mathbf{H}_i(t) = \mathbf{H}_i(t+\tau),\forall i = 0,1,\dots,m_s$的情况,我们称LDPC卷积码具有周期性。
若$\tau=1$,称LDPC卷积码为时不变的。
另外,截尾的LDPC卷积码的校验矩阵长度有限,即在$L$时刻终止,如
\begin{equation}
    \mathbf{H}_{[\infty]} = \left[
          \begin{array}{cccc}
            \mathbf{H}_0(1) & & & \\
            \mathbf{H}_1(1) & \mathbf{H}_0(2) & & \\
            \vdots & \mathbf{H}_1(2) & \ddots & \\
            \mathbf{H}_{m_s}(1) & \vdots & \ddots & \mathbf{H}_0(L) \\
             & \mathbf{H}_{m_s}(2) & \ddots & \mathbf{H}_1(L) \\
             & & \ddots & \vdots \\
             & & & \mathbf{H}_{m_s}(L)
          \end{array} \right]
\end{equation}

下面通过原模图来构造截尾的时不变的LDPC卷积码。
设原模图的基本矩阵为$(c-b)\times c$的$\mathbf{B}$。
首先构造$\mathbf{B}_{SC}$
\begin{equation}
    \mathbf{B}_{SC} = \left[
          \begin{array}{ccc}
            \mathbf{B}_0& & \\
            \mathbf{B}_1 & \mathbf{B}_0 & \\
            \vdots & \mathbf{B}_1 & \ddots \\
            \mathbf{B}_{m_s} & \vdots & \ddots \\
             & \mathbf{B}_{m_s} & \ddots \\
             & & \ddots 
          \end{array} \right]
\end{equation}
其中$m_s$在示意图中为当前原模图与前一个原模图相连的边数。$\mathbf{B}_0 , \mathbf{B}_1 , \dots , \mathbf{B}_{m_s}$为$(c-b)\times c$矩阵,且满足
\begin{equation}
\sum^{m_s}_{i=0} \mathbf{B}_i = \mathbf{B}
\end{equation}

然后将$\mathbf{B}_{SC}$中的非零元替换为随机选择的$M \times M$置换矩阵,将$\mathbf{B}_{SC}$中的零元替换为$M \times M$零矩阵,得到LDPC卷积码校验矩阵$\mathbf{H}_{SC}$。
最后将$\mathbf{H}_{SC}$中的非零元替换为从有限域GF($q$)中随机选取的元素,得到LDPC卷积码的$q$元校验矩阵$\mathbf{H}_{SC}$,其限制长度为$v_s=(m_s+1)Mc$。

图\ref{fig:LDPCCC}为通过原模图叠加方法构造LDPC卷积码的示意图。
首先复制几次(3,6)-正则LDPC码的原模图,设定记忆因子$m_s=1$,然后将原模图之间连接起来。
其中,原模图的基本矩阵为$ \mathbf{B} = [\begin{array}{cc} 3&3 \end{array}]$,使用到的组成矩阵为$\mathbf{B}_0 = [\begin{array}{cc} 2&1 \end{array}]$,$\mathbf{B}_1 = [\begin{array}{cc} 1&2 \end{array}]$。

\begin{center}
\def\linkdoub{\draw [double distance=1mm, very thick] (0,0)--}
\def\linksing{\draw [very thick] (0,0)--}
\def\check{%
    \filldraw [fill=white,very thick] (0,0) circle (5pt);
    \draw [very thick] (0,3.5pt)--(0,-3.5pt);
    \draw [very thick] (3.5pt,0)--(-3.5pt,0);
}
\def\bit{%
    \filldraw [fill=white,very thick] (0,0) circle (5pt);
    \draw [very thick] (-3.2pt,2.2pt)--(3.2pt,2.2pt);
    \draw [very thick] (-3.2pt,-2.2pt)--(3.2pt,-2.2pt);
}
\def\thetaone{90}
\def\thetatwo{-90}
\def\thetathree{60}
\def\thetafour{-60}
\def\armLength{0.9}
\def\symbolDist{1}
\begin{tikzpicture}
\foreach \x in {0.8,1.6,2.4,3.2,4}
{
\begin{scope}[shift=(0:\x)]
\linkdoub(\thetaone:\armLength);
\linksing(\thetaone:\armLength);
\linksing(\thetatwo:\armLength);
\begin{scope}[shift=(\thetaone:\symbolDist)]
\bit
\end{scope}
\begin{scope}[shift=(\thetatwo:\symbolDist)]
\linkdoub(\thetaone:\armLength);
\linksing(\thetaone:\armLength);
\bit
\end{scope}
\check
\end{scope}
}
\node at (4.7,0.9) {\ldots};
\node at (4.7,0) {\ldots};
\node at (4.7,-0.9) {\ldots};
\draw [->,very thick] (5.3,0) -- (6,0);
\begin{scope}[shift=(0:5.6)]
\foreach \x in {1,2,...,5}
{
	\begin{scope}[shift=(0:\x)]
		\linkdoub(\thetathree:\armLength);
		\linksing(\thetafour:\armLength);
		\check
		\begin{scope}[shift=(\thetathree:\symbolDist)]
		\linksing(\thetafour:\armLength);
		\bit
		\end{scope}
		\begin{scope}[shift=(\thetafour:\symbolDist)]
		\linkdoub(\thetathree:\armLength);
		\bit
		\end{scope}
	\end{scope}
}
\node at (6.5,0.9) {\ldots};
\node at (6.5,0) {\ldots};
\node at (6.5,-0.9) {\ldots};
\end{scope}
\end{tikzpicture}
\figurecaption{构造LDPC卷积码}
\label{fig:LDPCCC}
\end{center}

由于记忆参数$m_s=1$的LDPC卷积码有优秀的解码阈值,以及在有限长情况下,使用窗口参数较小的滑动窗口译码算法能达到较好的性能,
在之后的实证分析中,本文只考虑记忆参数$m_s=1$的LDPC卷积码。
并且本文只考虑正则$(d_v,d_c)$LDPC卷积码,即$\mathbf{H}_{SC}$中每行重量为$d_c$,每列重量为$d_v$。
这是因为正则$(d_v,d_c)$LDPC卷积码具有逼近信道容量的性能,以及其解码复杂度低于非正则LDPC卷积码。

为了公平地比较LDPC卷积码与LDPC分组码的性能,
本文限定构造LDPC卷积码与LDPC分组码的校验矩阵的方法如下。
选择两个$(c-b)\times c$矩阵$\mathbf{B}_0$和$\mathbf{B}_1$,使得$\mathbf{B}_0 + \mathbf{B}_1$为正则$(d_v,d_c)$矩阵。正则$(d_v,d_c)$LDPC分组码的基本矩阵为
\begin{equation}
    \mathbf{B}_{BC} = \left[
          \begin{array}{cc}
            \mathbf{B}_0 & \mathbf{B}_1\\
            \mathbf{B}_1 & \mathbf{B}_0
          \end{array} \right]_{2(c-b)\times 2c}
\end{equation}
其中$\mathbf{B}_{BC}$每个列向量的重量为$d_v$,每个行向量的重量为$d_c$。
然后使用第二节的原模图叠加方法构造LDPC分组码的校验矩阵
\begin{equation}
    \mathbf{H}_{BC} = \left[
          \begin{array}{cc}
            \mathbf{H}_0 & \mathbf{H}_1\\
            \mathbf{H}_1 & \mathbf{H}_0
          \end{array} \right]_{2(c-b)M \times 2cM}
\end{equation}

对于LDPC卷积码,使用相同的$\mathbf{B}_0$和$\mathbf{B}_1$,得到正则$(d_v,d_c)$LDPC卷积码的基本矩阵为
\begin{equation}
    \mathbf{B}_{CC} = \left[
          \begin{array}{ccccc}
            \mathbf{B}_0 & & & & \\
            \mathbf{B}_1 & \mathbf{B}_0 & & & \\
             & \mathbf{B}_1 & \mathbf{B}_0 & & \\
             & & \mathbf{B}_1 & \mathbf{B}_0 & \\
             & & & \mathbf{B}_1 & \ddots \\
             & & & & \ddots
          \end{array} \right]
\end{equation}
然后使用原模图叠加方法构造LDPC卷积码的校验矩阵
\begin{equation}
    \mathbf{H}_{CC} = \left[
          \begin{array}{ccccc}
            \mathbf{H}_0 & & & & \\
            \mathbf{H}_1 & \mathbf{H}_0 & & & \\
             & \mathbf{H}_1 & \mathbf{H}_0 & & \\
             & & \mathbf{H}_1 & \mathbf{H}_0 & \\
             & & & \mathbf{H}_1 & \ddots \\
             & & & & \ddots
          \end{array} \right]
\end{equation}
基于以上构造规则,可以使用其他的基本矩阵(见表\ref{table:formmatrix})构造LDPC分组码校验矩阵$\mathbf{H}_{BC}$与LDPC卷积码校验矩阵$\mathbf{H}_{CC}$。当原模图叠加数为$M$,有限域元素个数为$q=2^m$时,LDPC分组码的码长和LDPC卷积码的限制长度都等于$2Mmc$。

从LDPC卷积码的校验矩阵$\mathbf{H}_{CC}$以及LDPC分组码的校验矩阵$\mathbf{H}_{BC}$可以看出,$\mathbf{H}_{CC}$中的某一部分在周期性地重复$\mathbf{H}_{BC}$。
这种构造方式源于\parencite{782171}中,从LDPC分组码衍生LDPC卷积码的展开算法。
同时我们还可以注意到,尽管我们称$\mathbf{H}_{CC}$是正则$(d_v,d_c)$LDPC卷积码,
但是$\mathbf{H}_{CC}$并非真正的$(d_v,d_c)$正则校验矩阵,因为它的前$(c-b)$行的重量小于$d_c$。
而正则$(d_v,d_c)$LDPC卷积码的这种微小结构不规则性,正是其解码阈值接近信道容量的主要原因。
\begin{center}
\tablecaption{构造LDPC分组码和LDPC卷积码的组成矩阵}
\label{table:formmatrix}
\begin{tabular}{c|c|c}
 \hline
码 & 组成矩阵 &码长/限制长度 \\ \hline
(2,4)-正则 & 
$\mathbf{B}_0 = \mathbf{B}_1 = [\begin{array}{cc} 1&1 \end{array}]$ & 4Mm \\ \hline
(3,6)-正则 & 
 $\mathbf{B}_0 = [\begin{array}{cc} 2&1 \end{array}]$,$\mathbf{B}_1 = [\begin{array}{cc} 1&2 \end{array}]$ & 4Mm \\ \hline
\end{tabular}
\end{center}
\chapter{和积算法与滑动窗口解码的简介}
\section{LDPC分组码的和积算法}
和积算法是一种软判决算法。在算法迭代过程中,校验节点生成独立于信息节点接收到信息的额外信息,进而决定信息节点的值。以下介绍基本的和积算法。

记信源发送码字为$\mathbf{c}$,接收到的向量为$\mathbf{y}$。
将从校验节点$j$到它所连接信息节点$i$额外信息记为$E_{j,i}$。
如果某次迭代中,码字中$c_{i'}=1$的概率为$P_{j,i'}$,那么校验方程中包含奇数个$1$的概率为
\[
P_{j,i}^{ext} = \frac{1}{2} - \frac{1}{2} \prod_{i' \in B_j, i' \neq i} (1-2P_{j,i'})
\]
其中,$B_j$为与校验节点$j$相连的信息节点的下标集合。类似的,校验方程满足$c_{i}=0$时概率为$1-P_{j,i}^{ext}$。

校验节点$j$到信息节点$i$的额外信息用似然比来表示
\begin{eqnarray}
E_{j,i} & = & L(P_{j,i}^{ext})\\
& = & \text{log} \frac{1-P_{j,i}^{ext}}{P_{j,i}^{ext}}\\
& = & 2 \text{tanh}^{-1} \prod_{i' \in B_j, i' \neq i} \text{tanh} (M_{j,i'}/2)
\end{eqnarray}
其中
\[
M_{j,i'} \stackrel{\triangle}{=} L(P_{j,i'}) = \text{log} \frac{1-P_{j,i'}}{P_{j,i'}}
\]
每个信息节点接收输入的LLR,记为$R_i$
\[
R_i= \text{log} \frac{p(\mathbf{c}_i=0|y_i)}{p(\mathbf{c}_i=1|y_i)}
\]
同时接收来自相连接的校验节点的LLR。故信息节点$i$总的LLR为
\[
L_i = L(P_i) = R_i + \sum_{j\in A_i}E_{j,i}
\]
对于从信息节点发送到校验节点的信息,记为$M_{j,i}$
\[
M_{j,i}= R_i + \sum_{j' \in A_i,j'\neq j}E_{j',i}
\]
和积算法的具体步骤为
\begin{enumerate}
\item 给信息节点发送到校验节点的信息$M_{j,i}$赋值为$R_i$
\item 计算校验节点发送到信息节点的信息$E_{j,i}$
\item 计算信息节点的LLR,$L_i$。生成预测码字$\hat{c}$,代入校验方程,若满足,则停止算法。或者达到最大遍历值停止算法。
\item 计算信息节点发送到校验节点的信息$M_{j,i}$,遍历次数加一。继续第二步。
\end{enumerate}
如果该算法收敛,经过足够多次迭代后,将渐近求出码字中各位为1或者0的概率。实现逐符号最大后验概率译码。
\section{LDPC卷积码的滑动窗口解码}

\begin{center}
\def\linkdoub{\draw [double distance=1mm, very thick] (0,0)--}
\def\linksing{\draw [very thick] (0,0)--}
\def\check{%
    \filldraw [fill=white,very thick] (0,0) circle (5pt);
    \draw [very thick] (0,3.5pt)--(0,-3.5pt);
    \draw [very thick] (3.5pt,0)--(-3.5pt,0);
}
\def\bit{%
    \filldraw [fill=white,very thick] (0,0) circle (5pt);
    \draw [very thick] (-3.2pt,2.2pt)--(3.2pt,2.2pt);
    \draw [very thick] (-3.2pt,-2.2pt)--(3.2pt,-2.2pt);
}
\def\thetaone{60}
\def\thetatwo{-60}
\def\armLength{0.9}
\def\symbolDist{1}

\begin{tikzpicture}
\foreach \x in {1,2,...,5}
{
	\begin{scope}[shift=(0:\x)]
		\linkdoub(\thetaone:\armLength);
		\linksing(\thetatwo:\armLength);
		\check
		\begin{scope}[shift=(\thetaone:\symbolDist)]
		\linksing(\thetatwo:\armLength);
		\bit
		\end{scope}
		\begin{scope}[shift=(\thetatwo:\symbolDist)]
		\linkdoub(\thetaone:\armLength);
		\bit
		\end{scope}
	\end{scope}
}
\node at (6.5,0.9) {\ldots};
\node at (6.5,0) {\ldots};
\node at (6.5,-0.9) {\ldots};
\node [align=center] at (0,0) {Target\\
symbols
};
\draw [->] (0.3,0.4) -- (1.2,0.7);
\draw [->] (0.3,-0.4) -- (1.2,-0.7);
\draw [->] (0.3,-0.4) -- (1.2,-0.7);
\draw [dashed] (0.7,-1.3) -- (3.8,-1.3);
\draw [dashed] (0.7,1.3) -- (3.8,1.3);
\draw [dashed] (0.7,-1.3) -- (0.7,1.3);
\draw [dashed] (3.8,-1.3) -- (3.8,1.3);
\draw [thick] (0.7,1.3) -- (0.7,1.6);
\draw [thick] (1.8,1.3) -- (1.8,1.6);
\draw [thick] (2.8,1.3) -- (2.8,1.6);
\draw [thick] (3.8,1.3) -- (3.8,1.6);
\draw [<->,thick] (0.7,1.45) -- (1.8,1.45);
\node [align=center] at (1.25,1.65) {$Mc$};
\draw [thick] (0.7,-1.3) -- (0.7,-1.6);
\draw [thick] (3.8,-1.3) -- (3.8,-1.6);
\draw [<->,thick] (0.7,-1.45) -- (3.8,-1.45);
\node [align=center] at (2.25,-1.65) {$WMc$};
\end{tikzpicture}
\figurecaption{滑动窗口解码器的例子,$t=0$}

\begin{tikzpicture}
\foreach \x in {1,2,...,5}
{
	\begin{scope}[shift=(0:\x)]
		\linkdoub(\thetaone:\armLength);
		\linksing(\thetatwo:\armLength);
		\check
		\begin{scope}[shift=(\thetaone:\symbolDist)]
		\linksing(\thetatwo:\armLength);
		\bit
		\end{scope}
		\begin{scope}[shift=(\thetatwo:\symbolDist)]
		\linkdoub(\thetaone:\armLength);
		\bit
		\end{scope}
	\end{scope}
}
\node at (6.5,0.9) {\ldots};
\node at (6.5,0) {\ldots};
\node at (6.5,-0.9) {\ldots};
\node [align=center] at (0,0) {Decoded\\
symbols
};
\draw [->] (0.3,0.4) -- (1.2,0.7);
\draw [->] (0.3,-0.4) -- (1.2,-0.7);
\draw [->] (0.3,-0.4) -- (1.2,-0.7);
\draw [dashed] (1.8,-1.3) -- (4.8,-1.3);
\draw [dashed] (1.8,1.3) -- (4.8,1.3);
\draw [dashed] (1.8,-1.3) -- (1.8,1.3);
\draw [dashed] (4.8,-1.3) -- (4.8,1.3);
\draw [thick] (1.8,1.3) -- (1.8,1.6);
\draw [thick] (2.8,1.3) -- (2.8,1.6);
\draw [thick] (3.8,1.3) -- (3.8,1.6);
\draw [thick] (4.8,1.3) -- (4.8,1.6);
\draw [<->,thick] (1.8,1.45) -- (2.8,1.45);
\node [align=center] at (2.25,1.65) {$Mc$};
\draw [thick] (1.8,-1.3) -- (1.8,-1.6);
\draw [thick] (4.8,-1.3) -- (4.8,-1.6);
\draw [<->,thick] (1.8,-1.45) -- (4.8,-1.45);
\node [align=center] at (3.25,-1.65) {$WMc$};
\end{tikzpicture}
\figurecaption{滑动窗口解码器的例子,$t=1$}
\end{center}
上图为滑动窗口解码器的一个例子。其窗口大小为$W=3$,作用在$m_s=1$的(3,6)-正则$q$元LDPC卷积码上。
对于窗口里的$WMc$个符号,解码算法不断迭代,直到某一固定迭代次数或者满足某些停止规则。然后窗口平移$Mc$个位置,即已解码的$Mc$个符号从窗口移出。滑动窗口解码的解码延时为
\[
T_{CC} = WMmc
\]
因此,对于本文构造的LDPC分组码与LDPC卷积码,LDPC卷积码的解码延时$T_{CC}$等于LDPC分组码的码长,即等于LDPC分组码的解码延时$T_{BC}$。
滑动窗口解码器的窗口内迭代算法可以使用以上的和积算法,也可以用和积算法的改进版本\parencite{Barnault2003Fast}等。
窗口内迭代算法的停止规则如下。给定LDPC卷积码以及接收到的码字,令$P_t^{(j)}(b)$为在时刻$t$,窗口中第$j$个符号$v_t^{(j)}$的$b\in \text{GF} (q)$的概率。经过每次和积算法的迭代,基于$P_t^{(j)}(b)$对$v_t^{(j)}$进行硬判决,记为$\hat{v}_t^{(j)}$,即选择$\hat{v}_t^{(j)}=x$使得概率为最大值。记$e_t^{(j)}$为$v_t^{(j)}$不等于$\hat{v}_t^{(j)}$的概率,则
\[
e_t^{(j)}=1-P_t^{(j)}(x=\hat{v}_t^{(j)})
\]
而对于整个窗口内的符号,$e_t^{(j)}$的平均值为
\[
\hat{P}_t = \frac{1}{Mc}\sum_{j=0}^{Mc-1}e_t^{(j)}
\]
当迭代次数达到最大值$I_{max}$,或者$\hat{P}_t$小于某个选定的误符号率时,解码窗口才平移到下一个位置。
\chapter{模拟仿真结果}
本文模拟仿真的环境为翻转概率为$\epsilon$的BSC信道。
对于$q$元LDPC分组码,采用QSPA解码算法,停止规则为最大迭代次数$I_{max}=100$。
对于$q$元LDPC卷积码,滑动窗口解码使用QSPA解码算法,停止规则为最大迭代次数$I_{max}=100$或者误比特率BER达到$10^{-4}$。
\section{LDPC分组码码长与LDPC卷积码限制长度相等}
构造$q$元LDPC分组码,使其码长$n_{BC}=2Mc$。
构造$q$元LDPC卷积码,其限制长度$v_s=(m_s+1)Mc=2Mc$。
在$n_{BC}=v_s=2Mc$的条件下进行性能上的比较。
\begin{center}
\pgfplotsset{compat=1.13}
\begin{tikzpicture} 
\begin{axis}[
    xlabel={原模图复制数$M$},
    ylabel={所需翻转概率$\epsilon$},
    grid=major,
    legend style={
        font=\tiny,
        cells={anchor=east},
        legend pos=north west,
    },
]
\addplot[color=red,mark=square] table {data_d15.dat};
\addplot[color=red,mark=triangle] table {data_d16.dat};
\addplot[color=red,mark=o] table {data_d17.dat};
\addplot[color=blue,mark=square] table {data_d18.dat};
\addplot[color=blue,mark=triangle] table {data_d19.dat};
\addplot[color=blue,mark=o] table {data_d20.dat};
\legend{2-ary LDPC-BC,
        4-ary LDPC-BC,
        8-ary LDPC-BC,
        2-ary LDPC-CC,
        4-ary LDPC-CC,
        8-ary LDPC-CC}
\end{axis}
\end{tikzpicture}
\figurecaption{使用$R=1/2$的(2,4)正则原模图构造}
\label{fig:expone}
\end{center}

图\ref{fig:expone}展示的是$q$元LDPC分组码及$q$元LDPC卷积码解码时误比特率BER达到$10^{-4}$所需翻转概率$\epsilon$与原模图复制数$M$之间的关系。
$q$元LDPC分组码及$q$元LDPC卷积码都由码率为$R=1/2$的(2,4)正则原模图构造。
$q$元LDPC卷积码的滑窗解码器的窗口大小为$W=12$。

从图\ref{fig:expone}可以看出,$q$元LDPC分组码及$q$元LDPC卷积码的性能都随着原模图复制数$M$增大而提高。
同时复制数$M$较小时,相对于$q$元LDPC分组码,$q$元LDPC卷积码获得“卷积优势”,但是这种优势随$M$增大而减小。
比如当$M=100$时,8元LDPC卷积码所需的$\epsilon$比8元LDPC分组码的$\epsilon$大0.01,而$M=400$时,8元LDPC卷积码的$\epsilon$仅比8元LDPC分组码的$\epsilon$大0.001。

这个结果与LDPC卷积码阈值渐近分析(即$M$趋向无穷)相一致,也即LDPC卷积码的解码阈值会随着$M$的增大,逐渐趋向用于构建该LDPC卷积码的LDPC分组码的解码阈值。

\begin{center}
\pgfplotsset{compat=1.13}
\begin{tikzpicture} 
\begin{axis}[
    xlabel={原模图复制数$M$},
    ylabel={所需翻转概率$\epsilon$},
    grid=major,
    legend style={
        font=\tiny,
        cells={anchor=east},
        legend pos=north west,
    },
]
\addplot[color=red,mark=square] table {data_d21.dat};
\addplot[color=red,mark=triangle] table {data_d22.dat};
\addplot[color=red,mark=o] table {data_d23.dat};
\addplot[color=blue,mark=square] table {data_d24.dat};
\addplot[color=blue,mark=triangle] table {data_d25.dat};
\addplot[color=blue,mark=o] table {data_d26.dat};
\legend{2-ary LDPC-BC,
        4-ary LDPC-BC,
        8-ary LDPC-BC,
        2-ary LDPC-CC,
        4-ary LDPC-CC,
        8-ary LDPC-CC}
\end{axis}
\end{tikzpicture}
\figurecaption{使用$R=1/2$的(3,6)正则原模图构造}
\label{fig:exptwo}
\end{center}

图\ref{fig:exptwo}描述的是$q$元LDPC分组码及$q$元LDPC卷积码解码时误比特率BER达到$10^{-4}$所需翻转概率$\epsilon$与原模图复制数$M$之间的关系。
$q$元LDPC分组码及$q$元LDPC卷积码都由$R=1/2$的(3,6)正则原模图构造。
图\ref{fig:exptwo}与图\ref{fig:expone}类似,都表现了$q$元LDPC分组码及$q$元LDPC卷积码的性能随着$M$增大而提高的特性。
同时$q$元LDPC卷积码的“卷积优势”随$M$增大而减小。

另外,对比图\ref{fig:exptwo}与图\ref{fig:expone}可以发现,使用(3,6)正则原模图比使用(2,4)正则原模图构造LDPC码的解码性能更好。


\section{LDPC分组码与LDPC卷积码的解码延时相等}
除了解码性能以外,信道编码的解码延时对于设计高速通信系统来说十分重要。
无线通信如Wi-Fi,实时语音或视频通信,军事通信系统等对解码延时的要求比较苛刻。
下面考虑$q$元LDPC分组码解码延时等于$q$元LDPC卷积码解码延时的情况。
$q$元LDPC分组码的延时$T_{BC}=M_{BC}mc$,及$q$元LDPC卷积码的延时$T_{CC}=WM_{CC}mc$。
为使$T_{BC}=T_{CC}$,只需令$M_{BC} = WM_{CC}$。

\begin{center}
\pgfplotsset{compat=1.13}
\begin{tikzpicture}
\begin{semilogyaxis}[
    xlabel={反转概率$\epsilon$},
    ylabel={BER},
    grid=major,
    legend style={
        font=\tiny,
        cells={anchor=west},
        legend pos=south east,
    },
    ymin=0.00001,
    ymax=1,
]
\addplot[color=red,mark=square] table {data_d1.dat};
\addplot[color=red,mark=triangle] table {data_d2.dat};
\addplot[color=blue,mark=o] table {data_d3.dat};
\addplot[color=blue,mark=square] table {data_d4.dat};
\addplot[color=blue,mark=triangle] table {data_d5.dat};
\addplot[color=blue,mark=diamond] table {data_d6.dat};
\legend{LDPC-BC $M=192$\\
        LDPC-BC $M=384$\\
        LDPC-CC $M = 32,W =12$\\
        LDPC-CC $M = 64,W =6$\\
        LDPC-CC $M = 64,W =12$\\
        LDPC-CC $M = 128,W =6$\\}
\end{semilogyaxis}
\end{tikzpicture}
\figurecaption{使用$R=1/2$的(3,6)8元正则原模图构造}
\label{fig:perone}
\begin{tikzpicture} 
\begin{axis}[
    xlabel={解码延时(bits)},
    ylabel={所需翻转概率$\epsilon$},
    grid=major,
    legend style={
        font=\tiny,
        cells={anchor=west},
        legend pos=south east,
    },
    xmin=0,
]
\addplot table {data_d7.dat};
\addplot table {data_d8.dat};
\addplot table {data_d9.dat};
\addplot table {data_d10.dat};
\legend{LDPC-BC,$M = 32,64,128,192,288,384,576$\\
        LDPC-CC,$M = 32,W =2,3,\dots,12$\\
        LDPC-CC,$M = 64,W =2,3,\dots,12$\\
        LDPC-CC,$M = 128,W =2,3,\dots,8$\\}
\end{axis}
\end{tikzpicture}
\figurecaption{解码延时与所需翻转概率之间的关系}
\label{fig:pertwo}
\end{center}

图\ref{fig:perone}中8元LDPC分组码及8元LDPC卷积码都由码率为$R=1/2$的(3,6)正则原模图构造。
从图\ref{fig:perone}可以看出,8元LDPC卷积码的性能优于8元LDPC分组码。
另外,从图\ref{fig:perone}可以看出,用参数为$M_{CC}=64,W =6$解码器解码LDPC卷积码的性能优于用参数为$M_{CC}=32,W =12$解码LDPC卷积码的性能;
用参数为$M_{CC}=128,W =6$解码器解码LDPC卷积码的性能优于用参数为$M_{CC}=64,W =12$解码LDPC卷积码的性能(它们的解码延时相等)。
也就是说,选择相对小的窗口$W$,同时选择大的复制数$M_{CC}$补偿解码延时,能获得更好的解码性能。

图\ref{fig:pertwo}在GF($8$)上比较要达到相同的误比特率所需的翻转概率与解码延时之间的关系。
其中8元LDPC分组码及8元LDPC卷积码都由码率为$R=1/2$的(3,6)正则原模图构造。
从图\ref{fig:pertwo}可以看出,当固定原模图复制数$M_{CC}$时,LDPC卷积码性能随窗口大小$W$增加而提高,但是其解码延时$T_{CC}$却增加了。另外,在一定范围内大的复制数$M_{CC}$与小的窗口大小$W$的组合有更好的性能。
比如,解码延时$T_{CC}=2304$时,$M_{CC}=64$与$W=6$的LDPC卷积码的性能比$M_{CC}=32$与$W=12$的LDPC卷积码的性能更好。
然而,复制数$M_{CC}$越大与窗口$W$越小不一定有最好性能。比如,解码延时$T_{CC}=2304$时,$M_{CC}=128$与$W=3$的LDPC卷积码的性能反而比$M_{CC}=64$与$W=6$的LDPC卷积码的性能差。
由此可见,解码延时确定时,要综合考虑复制数$M_{CC}$与窗口大小$W$的影响。
\begin{center}
\tablecaption{在不同GF($q$)与解码延时的条件下使误比特率达到$10^{-4}$所需的翻转概率}
\label{table:filppingforbsc}
\begin{tabular}{|c|c|c|c|c|c|c|}
 \hline
解码 & \multicolumn{3}{|c|}{LDPC-BC} & \multicolumn{3}{|c|}{LDPC-CC} \\ \cline{2-7}
延时 & GF(2) & GF(4) & GF(8) & GF(2) & GF(4) & GF(8) \\ \hline
2304 bits & 0.14 & 0.15 & 0.15 & 0.12 & 0.16 & 0.18\\ \hline
4608 bits & 0.16 & 0.17 & 0.16 & 0.18 & 0.19 & 2.0\\ \hline
6912 bits & 0.17 & 0.18 & 0.17 & 0.19 & 0.21 & 0.21\\ \hline
9216 bits & 0.17 & 0.19 & 0.19 & 0.21 & 0.22 & 0.22\\ \hline
\end{tabular}
\end{center}

表\ref{table:filppingforbsc}展示了,(3,6)正则q元LDPC分组码与(3,6)正则q元LDPC卷积码在不同的GF($q$)以及不同的解码延时的条件下,误比特率BER达到$10^{-4}$所需翻转概率$\epsilon$。
从表\ref{table:filppingforbsc}可以看出,当解码延时确定时,非二元的LDPC卷积码的解码性能优于q元LDPC分组码以及二元LDPC卷积码的解码性能。
与q元LDPC分组码相反,q元LDPC卷积码误比特率BER达到$10^{-4}$所需翻转概率$\epsilon$随$q$的增加而增大。
这种现象与\parencite{6874959}中的迭代译码阈值结果相一致,即,当$q$增加时,(3,6)正则q元LDPC卷积码的解码阈值逐趋向到信道容量,而(3,6)正则q元LDPC分组码的解码阈值开始偏离信道容量。
我们还能从表中发现,对于解码延时为2304比特,(3,6)正则二元LDPC卷积码误比特率BER达到$10^{-4}$所需翻转概率$\epsilon$要低于,(3,6)正则二元LDPC分组码的翻转概率。
这是由于限制长度短的二元LDPC卷积码存在错误扩散现象。
在高误比特率或者更大解码延时的条件下,以上的错误扩散现象将会消失,
从表中也可以发现,当解码延时大于2304比特时,二元LDPC卷积码解码性能比二元LDPC分组码更好。



\section{LDPC分组码与LDPC卷积码的解码复杂度比较}
Pusane等人在\parencite{5695133}中分析了,相对于二元LDPC分组码,使用流水线解码算法的二元LDPC卷积码而获得的卷积特性与计算复杂度,处理器复杂度,解码器所需内存大小以及解码延时的关系。

在这一节中,我们将在相同解码延时或者相同的解码性能的条件下,比较q元LDPC卷积码与q元LDPC分组码的解码计算复杂度。
对于使用QSPA进行解码的q元LDPC码,在每次迭代周期中,一个校验节点所需的计算复杂度为$\mathcal{O}(qm)$,一个变量节点所需的计算复杂度为$\mathcal{O}(q)$。
令$I_{BC}$代表要解码一个LDPC分组码的码字所需要的平均迭代次数,类似的,令$I_{CC}$代表,在某一确定时间点,在LDPC卷积码的窗口解码器中一个窗口内,要解码所有符号所需要的平均迭代次数。
对于$(d_v,d_c)$正则LDPC分组码来说,其设计码率为$R=\frac{d_c-d_v}{d_c}$,其解码每个码字所需计算复杂度为
\begin{equation}
\mathcal{O} \left(\frac{T_{BC}}{m}d_vq+\frac{T_{BC}}{m}\left(1-R\right)d_cqm\right)I_{BC}=\mathcal{O}\left(\left(\frac{d_v}{m}+d_v\right)qT_{BC}\right)I_{BC}
\end{equation}
因此,对于解码$(d_v,d_c)$正则LDPC分组码的每一比特,所需计算复杂度为
\begin{equation}
\mathcal{O}\left(\left(\frac{d_v}{m}+d_v\right)q\right)I_{BC}
\end{equation}

对于$(d_v,d_c)$正则LDPC卷积码,为了简单起见,我们认为在窗口解码器的窗口中的那一部分Tanner图是$(d_v,d_c)$正则的,虽然在窗口起始部位的校验节点以及结束部位的变量节点有相对低的度分布。
因此,对于每个窗口,其计算复杂度大约为
\begin{equation}
\mathcal{O}\left(\left(\frac{d_v}{m}+d_v\right)qT_{CC}\right)I_{CC}
\end{equation}

而注意到在某一特定时间,窗口解码器要解码的比特数是$T_{CC}/W$,所以对于$(d_v,d_c)$正则LDPC卷积码每一个比特的解码计算复杂度是
\begin{equation}
\frac{\mathcal{O}\left(\left(\frac{d_v}{m}+d_v\right)qT_{CC}\right)I_{CC}}{T_{CC}/W} = \mathcal{O}\left(\left(\frac{d_v}{m}+d_v\right)q\right)WI_{CC}
\end{equation}

通过比较LDPC卷积码与LDPC分组码解码每一个比特所需的计算复杂度,我们可以发现,
如果$I_{BC}=WI_{CC}$,在相同的GF($q$)上,LDPC卷积码与LDPC分组码会有相同的解码计算复杂度。
在以下的实证分析中,我们所基于的原模图是(3,6)正则LDPC码。
具有其它度分布的原模图也能得出以下类似的实验结果。
另外,在解码LDPC卷积码时,我们将窗口参数设置为$W=6$。
\begin{center}
\tablecaption{在不同GF($q$)与解码延时的条件下使误比特率达到$10^{-4}$所需的迭代次数}
\label{table:iterationforbsc}
\begin{tabular}{|c|c|c|c|c|c|c|}
 \hline
解码 & \multicolumn{3}{|c|}{$I_{BC}$} & \multicolumn{3}{|c|}{$I_{CC}(W=6)$} \\ \cline{2-7}
延时 & GF(2) & GF(4) & GF(8) & GF(2) & GF(4) & GF(8) \\ \hline
4608 bits & 13.8 & 12.3 & 11.1 & 3.3 & 3.2 & 3.0\\ \hline
6912 bits & 15.6 & 14.1 & 12.6 & 3.9 & 3.7 & 3.4\\ \hline
\end{tabular}
\end{center}

在本节中,我们基于相等的解码延时的条件,比较q元LDPC卷积码与q元LDPC分组码的计算复杂度。
表\ref{table:iterationforbsc}展示了,在解码延时分别为4608,6912比特时,(3,6)正则q元LDPC卷积码与(3,6)正则q元LDPC分组码解码误比特率达到$10^{-4}$所需的平均迭代次数$I_{CC}$和$I_{BC}$。
我们可以发现,在相同的GF($q$)中,q元LDPC分组码的$I_{BC}$高于q元LDPC卷积码的$I_{CC}$。
这个现象的解释显而易见:对于给定的解码延时,LDPC分组码需要解码的符号数是LDPC卷积码需要解码的符号数的$W$倍。
我们还可以看到,对于LDPC分组码与LDPC卷积码,所需的迭代次数都随着$q$的减小而减小;
但是,总的解码计算复杂度却增加了,这是因为每次迭代的复杂度增加得更多。

\begin{center}
\begin{tikzpicture} 
\begin{axis}[
    xlabel={com com per bit},
    ylabel={GFQ},
    grid=major,
    legend style={
        font=\tiny,
        cells={anchor=west},
        legend pos=north west,
    },
    xmin=0,
    ymin=0,
]
\addplot table {data_d27.dat};
\addplot table {data_d28.dat};
\addplot table {data_d29.dat};
\addplot table {data_d30.dat};
\legend{4608 LDPC-BC\\
        6912 LDPC-BC\\
        4608 LDPC-CC\\
        6912 LDPC-CC\\}
\end{axis}
\end{tikzpicture}
\figurecaption{每解码一比特的计算复杂度与解码延时的关系}
\label{fig:perbit}
\end{center}

图\ref{fig:perbit}展示了,(3,6)正则q元LDPC卷积码与(3,6)正则q元LDPC分组码每解码一比特的计算复杂度与解码延时的关系。
我们可以观察到,LDPC卷积码与LDPC分组码的解码计算复杂度都随$q$的增大而指数增加。
另外,在相同的GF($q$)中,LDPC卷积码的复杂度比LDPC分组码的复杂度高35\%。
同时我们还可以看到,在相等解码延时的条件下,二元LDPC卷积码的复杂度比4元LDPC分组码的复杂度高10\%,而4元LDPC卷积码的复杂度比二元LDPC分组码的复杂度高80\%。
但是,二元LDPC卷积码解码时误比特率达到$10^{-4}$所需翻转概率,比4元LDPC分组码所需翻转概率大0.01,而4元LDPC卷积码所需翻转概率比二元LDPC分组码所需翻转概率大0.04(见表\ref{table:filppingforbsc})。
所以,即使LDPC卷积码相对于LDPC分组码有较高的计算复杂度,但是它性能的增加量比LDPC分组码更加明显。
况且,很难通过增加LDPC分组码的解码复杂度来增加其解码性能,也就是说,简单增加解码迭代次数,LDPC分组码也很难减小解码性能上与LDPC卷积码的差距。

因此,根据以上分析可以得出结论:对于给定解码延时的情况,在误码率性能和计算复杂度的平衡之间,LDPC卷积码是更好的选择。













\chapter{总结}
% -*- coding: utf-8 -*-

\def\bibrangedash{ $\sim$ }
\printbibliography [ category = cited]


% -*- coding: utf-8 -*-

%\makeschapterhead{致谢}
\chapter*{致谢}
本研究及学位论文是在陈鲁生导师的亲切关怀和悉心指导下完成的。他严肃的科学态度,严谨的治学精神,精益求精的工作作风,深深地感染和激励着我。从课题的选择到项目的最终完成,陈老师都始终给予我细心的指导和不懈的支持。这半年多来,陈老师不仅在学业上给我以精心指导,同时还在思想、生活上给我以无微不至的关怀,在此谨向陈老师致以诚挚的谢意和崇高的敬意。

在此,我还要感谢在一起愉快的度过本科四年生活各位同门,正是由于你们的帮助和支持,我才能克服一个一个的困难和疑惑,直至本文的顺利完成。

在论文即将完成之际,我的心情无法平静,从开始进入课题到论文的顺利完成,有多少可敬的师长、同学、朋友给了我无言的帮助,在这里请接受我诚挚的谢意!

最后我还要感谢培养我长大含辛茹苦的父母,谢谢你们!

\end{document}
