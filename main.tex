% -*- coding: utf-8 -*-
%%
%%
%%
%%
%%
%%
%%  本模板可以使用以下两种方式编译:
%%
%%     1. PDFLaTeX
%%
%%     2. XeLaTeX [推荐]
%%
%%  注意:
%%    1. 在改变编译方式前应先删除 *.toc 和 *.aux 文件,
%%       因为不同编译方式产生的辅助文件格式可能并不相同。
%%
%%
\documentclass[12pt,openright]{book}

\usepackage{ifxetex}
\ifxetex
  \usepackage[bookmarksnumbered]{hyperref}
\else
  \usepackage[unicode,bookmarksnumbered]{hyperref}
\fi

\usepackage[emptydoublepage]{NKThesis}   % 中文
%\usepackage[emptydoublepage,English]{NKThesis} % 英文

%   根据需要选择 biblatex 宏包选项.
\usepackage[
  backend = biber, style = caspervector, utf8,
  giveninits = true, sortgiveninits = true
]{biblatex}
\hypersetup{colorlinks=true,
            pdfborder=0 0 1,
            citecolor=black,
            linkcolor=black}
\usepackage{tikz}

\addbibresource{nkthesis.bib}
\DeclareBibliographyCategory{cited}
\AtEveryCitekey{\addtocategory{cited}{\thefield{entrykey}}}

\includeonly{
abstract,
manual,
tikz,
acknowledgements,
references,
appendices,
resume
}
\newtheorem{Theorem}{\hskip 2em 定理}[chapter]
\newtheorem{Lemma}[Theorem]{\hskip 2em 引理}
\newtheorem{Corollary}[Theorem]{\hskip 2em 推论}
\newtheorem{Proposition}[Theorem]{\hskip 2em 命题}
\newtheorem{Definition}[Theorem]{\hskip 2em 定义}
\newtheorem{Example}[Theorem]{\hskip 2em 例}
\begin{document}

%  设置基本信息
%  注意:  逗号`,'是项目分隔符. 如果某一项的值出现逗号, 应放在花括号内, 如 {,}
%
\NKTsetup{%
  论文题目(中文) = GF(q)上LDPC分组码和卷积码的比较,
  副标题         = ,
  论文题目(英文) = Comparison of LDPC Block and Convolutional Codes over GF(q),
  论文作者       = ,
  学号           = ,
  指导教师       = ,
  申请学位       = ,
  培养单位       = ,
  学科专业       = ,
  研究方向       = ,
  中图分类号     = ,
  UDC            = ,
  学校代码       = 10055,
  密级           = 公开,
                   % 公开 | 限制 | 秘密 | 机密, 若为公开, 不填以下三项
  保密期限       = ,
  审批表编号     = ,
  批准日期       = ,
  论文完成时间   = 二〇一六年五月,
  答辩日期       = ,
  论文类别       = 学士,
                   % 博士 | 学历硕士 | 硕士专业学位 | 高校教师 | 同等学力硕士
  院/系/所       = ,
  专业           = ,
  联系电话       = ,
  Email          = ,
  通讯地址(邮编) = ,
  备注           = }


% -*- coding: utf-8 -*-


\begin{zhaiyao}
本文介绍了GF(q)上LDPC分组码,并从LDPC分组码构建结构相似的卷积码。
接着分析了LDPC分组码的和积算法,以及用于LDPC卷积码的滑动窗口和积算法。
在不同参数的GF(q)上,本文主要考虑了两种情况:一种是LDPC分组码解码延时与LDPC卷积码的解码延时相等的情况;另一种是LDPC分组码码长与LDPC卷积码的限制长度相等的情况。基于以上条件比较多元LDPC分组码与二元、多元LDPC卷积码的解码性能。
其中模拟的结果表明q元LDPC卷积码的解码性能好与2元LDPC卷积码,以及q元LDPC卷积码的解码性能好与q元LDPC分组码。
\end{zhaiyao}


\begin{guanjianci}
LDPC分组码;LDPC卷积码;滑动窗口;和积算法;解码延时
\end{guanjianci}



\begin{abstract}
In this paper, we introduce LDPC block codes (LDPC-BC), and LDPC convolutional codes (LDPC-CC) which are derived form the former. Then we analyse sum product algorithm(SPA) for LDPC-BC, and a sliding window decoder(WD) for LDPC-CC. Base on different GF(q), we compare the decoding performance between q-ary LDPC-BC and LDPC-CC in two regimes: one when the constraint length of q-ary SC-LDPC codes is equal to the block length of q-ary LDPC-BCs and the other when the two decoding latencies are equal. Simulation shows that q-ary LDPC-CC outperform binary LDPC-CC and q-ary LDPC-BC.
\end{abstract}


\begin{keywords}
LDPC block codes; LDPC convolutional codes; sliding window decoder; belief propagation; decoding latencie
\end{keywords} 
\tableofcontents
%\include{manual}
\chapter{背景及介绍}
Gallager在1962年提出低密度校验分组码(引文),同时他还提出了一种信息遍历解码算法。
在(引文)中,模拟结果显示,用信息遍历算法解码LDPC分组码的性能超过了turbo码,并接近香农极限。
在Low-density parity-check code over GF(q)1998,Davey和MacKay考虑了定义在GF(q)的LDPC分组码并且把Gallager的BP解码算法推广到q元,称作和积算法,并得到很好的性能。

LDPC卷积码是由Felström和Zigangirov在(引文)提出来的。
我们可以通过将几个正则LDPC码叠加连接在一起形成LDPC卷积码。
LDPC卷积码的特点是它的BP解码阈值能达到它所基于的LDPC码的最大后验概率(MAP)解码阈值(引文)。
为了解决LDPC卷积码顺序解码时占用大量内存和解码延时过长的缺点,在(引文)Windowed erasure decoding of LDPC convolutional codes中考虑了一种窗口解码,并研究了解码延时和解码性能的取舍。

由于LDPC卷积码与LDPC分组码具有相似的结构,并且在解码延时和解码性能中的取舍,一些文章已经开始研究其二者的性能比较Comparison of LDPC Block and LDPC Convolutional Codes Based on their Decoding Latency,LDPC Codes and Convolutional Codes with Equal Structural Delay: A Comparison。
本文基于以上进行q元域的推广,同时比较别的实用参数。
本文的结构如下:第一章主要是LDPC卷积码与LDPC分组码背景及介绍。第二章主要是描述LDPC卷积码与LDPC分组码的构建过程,要在特殊的构建原则上两者才能比较。第三章主要描述LDPC卷积码与LDPC分组码的解码算法,并且计算了两种算法的解码延时。第四章通过固定几个参数比较另外的参数,比较LDPC卷积码与LDPC分组码的性能。第五章是结论及总结。
\chapter{基于原型构造的LDPC码}
\section{LDPC码的原型}
每一个LDPC码都对应着一个Tanner图,也可称为原型,或者称为基本图。如下图,
\begin{center}
\begin{tikzpicture}
[check/.style={rectangle,fill=black!50,thick},
bit/.style={circle,fill=black!50,thick}]
\draw (-3,4) -- (-2,0);
\draw (3,4) -- (-2,0);
\draw (-1,4) -- (0,0);
\draw (1,4) -- (0,0);
\draw (-3,4) -- (2,0);
\draw (-1,4) -- (2,0);
\draw (1,4) -- (2,0);
\draw (3,4) -- (2,0);
\node[bit] at ( 1,4) [/tikz/label=above:3] {}; 
\node[bit] at (-1,4) [/tikz/label=above:2] {};
\node[bit] at (-3,4) [/tikz/label=above:1] {};
\node[bit] at ( 3,4) [/tikz/label=above:4] {}; 
\node[check] at ( 0,0) [/tikz/label=below:2] {}; 
\node[check] at ( 2,0) [/tikz/label=below:3] {}; 
\node[check] at (-2,0) [/tikz/label=below:1] {};
\end{tikzpicture}
\figurecaption{基本图}
\end{center}
该基本图描述的LDPC码的校验矩阵记为
\begin{equation}
    H_b = \left(
      \begin{array}{cccc}
        1 & 0 & 0 & 1 \\
        0 & 1 & 1 & 0 \\
        1 & 1 & 1 & 1 
      \end{array} \right)
\end{equation}
可以通过叠加基本图构造有结构的LDPC码。
比如,先复制m次基本图,得到簇状的Tanner图,如下图。
\begin{center}
\begin{tikzpicture}
[bit/.style={circle,fill=black!50,thick},
check/.style={rectangle,fill=black!50,thick}]
\foreach \x / \y / \z in {0/1/0,0.4/2/0.1,0.8/3/0.2,1.2/4/0.3,1.6/5/0.4}
{
	\draw[shift={(\x,\z)}] (-4,4) -- (-3,0);
	\draw[shift={(\x,\z)}] (4,4) -- (-3,0);
	\draw[shift={(\x,\z)}] (-2,4) -- (0,0);
	\draw[shift={(\x,\z)}] (2,4) -- (0,0);
	\draw[shift={(\x,\z)}] (-4,4) -- (3,0);
	\draw[shift={(\x,\z)}] (-2,4) -- (3,0);
	\draw[shift={(\x,\z)}] (2,4) -- (3,0);
	\draw[shift={(\x,\z)}] (4,4) -- (3,0);
	\node[bit,shift={(\x,\z)}] at ( 2,4) [/tikz/label=above:3\y] {}; 
	\node[bit,shift={(\x,\z)}] at (-2,4) [/tikz/label=above:2\y] {};
	\node[bit,shift={(\x,\z)}] at (-4,4) [/tikz/label=above:1\y] {};
	\node[bit,shift={(\x,\z)}] at ( 4,4) [/tikz/label=above:4\y] {}; 
	\node[check,shift={(\x,\z)}] at ( 0,0) [/tikz/label=below:2\y] {}; 
	\node[check,shift={(\x,\z)}] at ( 3,0) [/tikz/label=below:3\y] {}; 
	\node[check,shift={(\x,\z)}] at (-3,0) [/tikz/label=below:1\y] {};
}
\end{tikzpicture}
\end{center}
然后将基本图对应的$H_b$中为$1$的元素换为来自置换矩阵集合的元素。即
\begin{equation}
    H = \left(
      \begin{array}{cccc}
        P^2 & 0 & 0 & P^5 \\
        0 & P^5 & P^1 & 0 \\
        P^4 & P^3 & P^2 & P^4 
      \end{array} \right)
\end{equation}
其中$p^i$对应循环置换矩阵,比如
\begin{equation}
    P^2 = \left(
      \begin{array}{ccccc}
        0 & 0 & 0 & 0 & 1 \\
        1 & 0 & 0 & 0 & 0 \\
        0 & 1 & 0 & 0 & 0 \\
        0 & 0 & 1 & 0 & 0 \\
        0 & 0 & 0 & 1 & 0 
      \end{array} \right)
\end{equation}
此时构造出来的LDPC码的Tanner图如下
\begin{center}
\begin{tikzpicture}
[bit/.style={circle,fill=black!50,thick},
check/.style={rectangle,fill=black!50,thick}]
%p2
	\draw (-2.4,4.4) -- (-3,0);
	\draw (-4,4) -- (-2.6,0.1);
	\draw (-3.6,4.1) -- (-2.2,0.2);
	\draw (-3.2,4.2) -- (-1.8,0.3);
	\draw (-2.8,4.3) -- (-1.4,0.4);
%p5
	\draw (4.4,4.1) -- (-3,0);
	\draw (4.8,4.2) -- (-2.6,0.1);
	\draw (5.2,4.3) -- (-2.2,0.2);
	\draw (5.6,4.4) -- (-1.8,0.3);
	\draw (4,4) -- (-1.4,0.4);
%p5
	\draw (-1.6,4.1) -- (0,0);
	\draw (-1.2,4.2) -- (0.4,0.1);
	\draw (-0.8,4.3) -- (0.8,0.2);
	\draw (-0.4,4.4) -- (1.2,0.3);
	\draw (-2,4) -- (1.6,0.4);
%p4
	\draw (-3.2,4.2) -- (3,0);
	\draw (-2.8,4.3) -- (3.4,0.1);
	\draw (-2.4,4.4) -- (3.8,0.2);
	\draw (-4,4)   -- (4.2,0.3);
	\draw (-3.6,4.1) -- (4.6,0.4);
%p3
	\draw (-0.8,4.3) -- (3,0);
	\draw (-0.4,4.4) -- (3.4,0.1);
	\draw (-2,4)   -- (3.8,0.2);
	\draw (-1.6,4.1) -- (4.2,0.3);
	\draw (-1.2,4.2) -- (4.6,0.4);
%p2
	\draw (3.6,4.4) -- (3,0);
	\draw (2,4)   -- (3.4,0.1);
	\draw (2.4,4.1) -- (3.8,0.2);
	\draw (2.8,4.2) -- (4.2,0.3);
	\draw (3.2,4.3) -- (4.6,0.4);
%p4
	\draw (4.8,4.2) -- (3,0);
	\draw (5.2,4.3) -- (3.4,0.1);
	\draw (5.6,4.4) -- (3.8,0.2);
	\draw (4,4)   -- (4.2,0.3);
	\draw (4.4,4.1) -- (4.6,0.4);
\foreach \x / \y / \z in {0/1/0,0.4/2/0.1,0.8/3/0.2,1.2/4/0.3,1.6/5/0.4}
{
	\draw[shift={(\x,\z)}] (2,4) -- (0,0);
	\node[bit,shift={(\x,\z)}] at ( 2,4) [/tikz/label=above:3\y] {}; 
	\node[bit,shift={(\x,\z)}] at (-2,4) [/tikz/label=above:2\y] {};
	\node[bit,shift={(\x,\z)}] at (-4,4) [/tikz/label=above:1\y] {};
	\node[bit,shift={(\x,\z)}] at ( 4,4) [/tikz/label=above:4\y] {}; 
	\node[check,shift={(\x,\z)}] at ( 0,0) [/tikz/label=below:2\y] {}; 
	\node[check,shift={(\x,\z)}] at ( 3,0) [/tikz/label=below:3\y] {}; 
	\node[check,shift={(\x,\z)}] at (-3,0) [/tikz/label=below:1\y] {};
}
\end{tikzpicture}
\end{center}
\section{LDPC分组码的构造}
设计码率为$R=b/c$的LDPC分组码的原型是一个二分图,有$c$个变量节点和$c-b$个校验节点。
它能生成不同码长的,设计码率为$R$,有相同的度分布的分组码。
GF($q$)为含$q=2^m$个元素的有限域,其中$m$为在GF($q$)代表一个符号所需的位数。令$M$为原型叠加数。通过以下两步从原型的$(c-b)\times c$邻接矩阵$\mathbf{B}=[B_{i,j}]$构造码长为$n_{BC}=Mc$的$q$元LDPC分组码:
\begin{enumerate}
\item 将$\mathbf{B}$中的非零元$B_{i,j}$替换为随机选择的$M \times M$置换矩阵,将$\mathbf{B}$中的零元$B_{i,j}$替换为$M \times M$零矩阵。此时得到对应于$\mathbf{B}$的二元校验矩阵$\mathbf{H}$;
\item 将$\mathbf{H}$中非零元替换为从有限域GF($q$)中随机选取的元素,得到LDPC分组码的$q$元校验矩阵$\mathbf{H}_{BC}$。
\end{enumerate}
对于LDPC分组码,必须等待整个码块接受完毕才能执行置信传播解码算法。故$q$元LDPC分组码的解码延时为
\[
T_{BC}=n_{BC}\cdot m = Mmc
\]
\section{LDPC卷积码的构造}
LDPC卷积码也可以通过叠加原型来构造。
原型的基本矩阵为$(c-b)\times c$的$\mathbf{B}$,以此构造码率为$R=b/c$的LDPC卷积码。首先构造$\mathbf{B}_{SC}$
\begin{equation}
    \mathbf{B}_{SC} = \left[
          \begin{array}{ccc}
            \mathbf{B}_0& & \\
            \mathbf{B}_1 & \mathbf{B}_0 & \\
            \vdots & \mathbf{B}_1 & \ddots \\
            \mathbf{B}_{m_s} & \vdots & \ddots \\
             & \mathbf{B}_{m_s} & \ddots \\
             & & \ddots 
          \end{array} \right]
\end{equation}
其中$m_s$为记忆因子,即当前原型与前一个原型相连的边数。$\mathbf{B}_0 , \mathbf{B}_1 , \dots , \mathbf{B}_{m_s}$为$(c-b)\times c$矩阵,且满足
\[
\sum^{m_s}_{i=0} \mathbf{B}_i = \mathbf{B}
\]
然后将$\mathbf{B}_{SC}$中的非零元替换为随机选择的$M \times M$置换矩阵,将$\mathbf{B}_{SC}$中的零元替换为$M \times M$零矩阵,得到LDPC卷积码校验矩阵$\mathbf{H}_{SC}$。
最后将$\mathbf{H}_{SC}$中的非零元替换为从有限域GF($q$)中随机选取的元素,得到LDPC卷积码的$q$元校验矩阵$\mathbf{H}_{BC}$,其限制长度(决定了非零对角带的最大宽度),为$v_s=(m_s+1)Mc$。

本文为简单起见,采用$m_s=1$,并且只考虑正则$(d_v,d_c)$LDPC卷积码,即$\mathbf{H}_{SC}$中每行重量为$d_c$,每列重量为$d_v$。
同时限定随机选择置换矩阵的方法如下。
选择两个$(c-b)\times c$矩阵$\mathbf{B}_0$和$\mathbf{B}_1$,使得$\mathbf{B}_0 + \mathbf{B}_1$为正则$(d_v,d_c)$矩阵。LDPC分组码的基本矩阵为
\begin{equation}
    \mathbf{B}_{BC} = \left[
          \begin{array}{cc}
            \mathbf{B}_0 & \mathbf{B}_1\\
            \mathbf{B}_1 & \mathbf{B}_0
          \end{array} \right]_{2(c-b)\times 2c}
\end{equation}
然后使用第二节的原型叠加方法构造LDPC分组码的校验矩阵
\begin{equation}
    \mathbf{H}_{BC} = \left[
          \begin{array}{cc}
            \mathbf{H}_0 & \mathbf{H}_1\\
            \mathbf{H}_1 & \mathbf{H}_0
          \end{array} \right]_{2(c-b)M \times 2cM}
\end{equation}
对于LDPC卷积码,使用相同的$\mathbf{B}_0$和$\mathbf{B}_1$,得到正则$(d_v,d_c)$LDPC卷积码的基本矩阵为
\begin{equation}
    \mathbf{B}_{SC} = \left[
          \begin{array}{ccccc}
            \mathbf{B}_0 & & & & \\
            \mathbf{B}_1 & \mathbf{B}_0 & & & \\
             & \mathbf{B}_1 & \mathbf{B}_0 & & \\
             & & \mathbf{B}_1 & \mathbf{B}_0 & \\
             & & & \mathbf{B}_1 & \ddots \\
             & & & & \ddots
          \end{array} \right]
\end{equation}
然后使用原型叠加方法构造LDPC卷积码的校验矩阵
\begin{equation}
    \mathbf{H}_{SC} = \left[
          \begin{array}{ccccc}
            \mathbf{H}_0 & & & & \\
            \mathbf{H}_1 & \mathbf{H}_0 & & & \\
             & \mathbf{H}_1 & \mathbf{H}_0 & & \\
             & & \mathbf{H}_1 & \mathbf{H}_0 & \\
             & & & \mathbf{H}_1 & \ddots \\
             & & & & \ddots
          \end{array} \right]
\end{equation}
基于以上构造规则,可以使用其他的基本矩阵(见下表)构造LDPC分组码校验矩阵$\mathbf{H}_{BC}$与LDPC卷积码校验矩阵$\mathbf{H}_{SC}$。当原型叠加数为$M$,有限域元素个数为$q=2^m$时,LDPC分组码的码长和LDPC卷积码的限制长度都等于$2Mmc$。
\begin{center}
\tablecaption{构造LDPC分组码和LDPC卷积码的组成矩阵}
\begin{tabular}{c|c|c}
 \hline
码 & 组成矩阵 &码长/限制长度 \\ \hline
(2,4)-正则 & 
$\mathbf{B}_0 = \mathbf{B}_1 = [\begin{array}{cc} 1&1 \end{array}]$ & 4Mm \\ \hline
(3,6)-正则 & 
 $\mathbf{B}_0 = [\begin{array}{cc} 2&1 \end{array}]$,$\mathbf{B}_1 = [\begin{array}{cc} 1&2 \end{array}]$ & 4Mm \\ \hline
(3,9)-正则 & 
 $\mathbf{B}_0 = [\begin{array}{ccc} 1&2&2 \end{array}]$,$\mathbf{B}_1 = [\begin{array}{ccc} 2&2&1 \end{array}]$ & 6Mm \\ \hline
(3,12)-正则 & 
 $\mathbf{B}_0 = [\begin{array}{cccc} 1&1&2&2 \end{array}]$,$\mathbf{B}_1 = [\begin{array}{cccc} 2&2&1&1 \end{array}]$ & 8Mm \\ \hline

\end{tabular}
\end{center}
\chapter{和积算法与滑动窗口解码的简介}
\section{LDPC分组码的和积算法}
和积算法是一种软判决算法。在算法迭代过程中,校验节点生成独立于信息节点接收到信息的额外信息,进而决定信息节点的值。以下介绍基本的和积算法。

记信源发送码字为$\mathbf{c}$,接收到的向量为$\mathbf{y}$。
将从校验节点$j$到它所连接信息节点$i$额外信息记为$E_{j,i}$。
如果某次迭代中,码字中$\mathbf{c}_{i'}=1$的概率为$P_{j,i'}$,那么校验方程中包含奇数个$1$的概率为
\begin{equation}
P_{j,i}^{ext} = \frac{1}{2} - \frac{1}{2} \prod_{i' \in B_j, i' \neq i} (1-2P_{j,i'})
\end{equation}

其中,$B_j$为与校验节点$j$相连的信息节点的下标集合。类似的,校验方程满足$\mathbf{c}_{i}=0$时概率为$1-P_{j,i}^{ext}$。

每个信息节点接收输入的LLR,记为$R_i$,若信道为对称信道,即$p(\mathbf{c}_i=0) = p(\mathbf{c}_i=1)$
\begin{equation}
R_i= \text{log} \frac{p(\mathbf{c}_i=0|y_i)}{p(\mathbf{c}_i=1|y_i)} = \text{log} \frac{p(y_i|\mathbf{c}_i=0)}{p(y_i|\mathbf{c}_i=1)}
\end{equation}

对于不同的信道模型以及调制方式,$R_i$的具体形式都是不同的。
对于本文使用翻转概率为$p$的BSC信道而言
\begin{equation}
  R_i = \left \{
    \begin{array}{rl}
      \frac{1-p}{p}, & y_i=0 \\
      \frac{p}{1-p}, & y_i=1 
    \end{array} \right.
\end{equation}

对于AWGN信道,BPSK调制方式而言,调制后的信号幅度为$s=\{\sqrt{E_b},-\sqrt{E_b}\}$,对应$\mathbf{c}_i=0$和$\mathbf{c}_i=1$,则
\begin{equation}
R_i= \text{log} \frac{ \text{exp}(-\frac{1}{2\sigma^2}(x+\sqrt{E_b})^2) }{\text{exp}(-\frac{1}{2\sigma^2}(x-\sqrt{E_b})^2)} = -\frac{2\sqrt{E_b}}{\sigma^2}\mathbf{c}_i
\end{equation}

校验节点$j$到信息节点$i$的额外信息用似然比来表示
\begin{eqnarray}
E_{j,i} & = & L(P_{j,i}^{ext})\\
& = & \text{log} \frac{1-P_{j,i}^{ext}}{P_{j,i}^{ext}}\\
& = & 2 \text{tanh}^{-1} \prod_{i' \in B_j, i' \neq i} \text{tanh} (M_{j,i'}/2)
\end{eqnarray}
其中
\begin{equation}
M_{j,i'} \stackrel{\triangle}{=} L(P_{j,i'}) = \text{log} \frac{1-P_{j,i'}}{P_{j,i'}}
\end{equation}

信息节点$i$除了接收输入的LLR以外,还接收来自相连接的校验节点的LLR。故信息节点$i$总的LLR为
\begin{equation}
L_i = L(P_i) = R_i + \sum_{j\in A_i}E_{j,i}
\end{equation}
对于从信息节点发送到校验节点的信息,记为$M_{j,i}$
\begin{equation}
M_{j,i}= R_i + \sum_{j' \in A_i,j'\neq j}E_{j',i}
\end{equation}

故和积算法的具体步骤为
\begin{enumerate}
\item 给信息节点发送到校验节点的信息$M_{j,i}$赋值为$R_i$
\item 计算校验节点发送到信息节点的信息$E_{j,i}$
\item 计算信息节点的LLR,$L_i$。生成预测码字$\hat{c}$,代入校验方程,若满足,则停止算法。或者达到最大遍历值停止算法。
\item 计算信息节点发送到校验节点的信息$M_{j,i}$,遍历次数加一。继续第二步。
\end{enumerate}
如果该算法收敛,经过足够多次迭代后,将渐近求出码字中各位为1或者0的概率,进而实现逐符号最大后验概率译码。

\section{LDPC卷积码的滑动窗口解码}

\begin{center}
\def\linkdoub{\draw [double distance=1mm, very thick] (0,0)--}
\def\linksing{\draw [very thick] (0,0)--}
\def\check{%
    \filldraw [fill=white,very thick] (0,0) circle (5pt);
    \draw [very thick] (0,3.5pt)--(0,-3.5pt);
    \draw [very thick] (3.5pt,0)--(-3.5pt,0);
}
\def\bit{%
    \filldraw [fill=white,very thick] (0,0) circle (5pt);
    \draw [very thick] (-3.2pt,2.2pt)--(3.2pt,2.2pt);
    \draw [very thick] (-3.2pt,-2.2pt)--(3.2pt,-2.2pt);
}
\def\thetaone{60}
\def\thetatwo{-60}
\def\armLength{0.9}
\def\symbolDist{1}

\begin{tikzpicture}
\foreach \x in {1,2,...,5}
{
	\begin{scope}[shift=(0:\x)]
		\linkdoub(\thetaone:\armLength);
		\linksing(\thetatwo:\armLength);
		\check
		\begin{scope}[shift=(\thetaone:\symbolDist)]
		\linksing(\thetatwo:\armLength);
		\bit
		\end{scope}
		\begin{scope}[shift=(\thetatwo:\symbolDist)]
		\linkdoub(\thetaone:\armLength);
		\bit
		\end{scope}
	\end{scope}
}
\node at (6.5,0.9) {\ldots};
\node at (6.5,0) {\ldots};
\node at (6.5,-0.9) {\ldots};
\node [align=center] at (0,0) {Target\\
symbols
};
\draw [->] (0.3,0.4) -- (1.2,0.7);
\draw [->] (0.3,-0.4) -- (1.2,-0.7);
\draw [->] (0.3,-0.4) -- (1.2,-0.7);
\draw [dashed] (0.7,-1.3) -- (3.8,-1.3);
\draw [dashed] (0.7,1.3) -- (3.8,1.3);
\draw [dashed] (0.7,-1.3) -- (0.7,1.3);
\draw [dashed] (3.8,-1.3) -- (3.8,1.3);
\draw [thick] (0.7,1.3) -- (0.7,1.6);
\draw [thick] (1.8,1.3) -- (1.8,1.6);
\draw [thick] (2.8,1.3) -- (2.8,1.6);
\draw [thick] (3.8,1.3) -- (3.8,1.6);
\draw [<->,thick] (0.7,1.45) -- (1.8,1.45);
\node [align=center] at (1.25,1.65) {$Mc$};
\draw [thick] (0.7,-1.3) -- (0.7,-1.6);
\draw [thick] (3.8,-1.3) -- (3.8,-1.6);
\draw [<->,thick] (0.7,-1.45) -- (3.8,-1.45);
\node [align=center] at (2.25,-1.65) {$WMc$};
\end{tikzpicture}
\figurecaption{滑动窗口解码器的例子,$t=0$}
\label{fig:windowzero}

\begin{tikzpicture}
\foreach \x in {1,2,...,5}
{
	\begin{scope}[shift=(0:\x)]
		\linkdoub(\thetaone:\armLength);
		\linksing(\thetatwo:\armLength);
		\check
		\begin{scope}[shift=(\thetaone:\symbolDist)]
		\linksing(\thetatwo:\armLength);
		\bit
		\end{scope}
		\begin{scope}[shift=(\thetatwo:\symbolDist)]
		\linkdoub(\thetaone:\armLength);
		\bit
		\end{scope}
	\end{scope}
}
\node at (6.5,0.9) {\ldots};
\node at (6.5,0) {\ldots};
\node at (6.5,-0.9) {\ldots};
\node [align=center] at (0,0) {Decoded\\
symbols
};
\draw [->] (0.3,0.4) -- (1.2,0.7);
\draw [->] (0.3,-0.4) -- (1.2,-0.7);
\draw [->] (0.3,-0.4) -- (1.2,-0.7);
\draw [dashed] (1.8,-1.3) -- (4.8,-1.3);
\draw [dashed] (1.8,1.3) -- (4.8,1.3);
\draw [dashed] (1.8,-1.3) -- (1.8,1.3);
\draw [dashed] (4.8,-1.3) -- (4.8,1.3);
\draw [thick] (1.8,1.3) -- (1.8,1.6);
\draw [thick] (2.8,1.3) -- (2.8,1.6);
\draw [thick] (3.8,1.3) -- (3.8,1.6);
\draw [thick] (4.8,1.3) -- (4.8,1.6);
\draw [<->,thick] (1.8,1.45) -- (2.8,1.45);
\node [align=center] at (2.25,1.65) {$Mc$};
\draw [thick] (1.8,-1.3) -- (1.8,-1.6);
\draw [thick] (4.8,-1.3) -- (4.8,-1.6);
\draw [<->,thick] (1.8,-1.45) -- (4.8,-1.45);
\node [align=center] at (3.25,-1.65) {$WMc$};
\end{tikzpicture}
\figurecaption{滑动窗口解码器的例子,$t=1$}
\label{fig:windowone}
\end{center}

图\ref{fig:windowzero}与图\ref{fig:windowone}为滑动窗口解码器的一个例子。
其窗口大小为$W=3$,作用在$m_s=1$的(3,6)正则$q$元LDPC卷积码上。
对于窗口里的$WMc$个符号,解码算法不断迭代,直到某一固定迭代次数或者满足某些停止规则。然后窗口平移$Mc$个位置,即已解码的$Mc$个符号从窗口移出。滑动窗口解码的解码延时为
\begin{equation}
T_{CC} = WMmc
\end{equation}

本文基于,LDPC卷积码的解码延时$T_{CC}$等于LDPC分组码的解码延时$T_{BC}$,的前提上,研究LDPC卷积码及LDPC分组码的性能。
滑动窗口解码器的窗口内迭代算法可以使用之前描述的和积算法,也可以用和积算法的改进版本FFT-QSPA等。

窗口内迭代算法的停止规则如下。给定LDPC卷积码以及接收到的码字,令$P_t^{(j)}(b)$为在时刻$t$,窗口中第$j$个符号$v_t^{(j)}$的$b\in \text{GF} (q)$的概率。
经过每次和积算法的迭代,基于$P_t^{(j)}(b)$对$v_t^{(j)}$进行硬判决,记为$\hat{v}_t^{(j)}$,即选择$\hat{v}_t^{(j)}=x$使得概率为最大值。
记$e_t^{(j)}$为$v_t^{(j)}$不等于$\hat{v}_t^{(j)}$的概率,则
\begin{equation}
e_t^{(j)}=1-P_t^{(j)}(x=\hat{v}_t^{(j)})
\end{equation}
而对于整个窗口内的符号,$e_t^{(j)}$的平均值为
\begin{equation}
\hat{P}_t = \frac{1}{Mc}\sum_{j=0}^{Mc-1}e_t^{(j)}
\end{equation}
当迭代次数达到最大值$I_{max}$,或者$\hat{P}_t$小于某个选定的误符号率时,解码窗口才平移到下一个位置。
\chapter{模拟仿真结果}
本文模拟仿真的环境为BI-AWGN信道中使用BPSK调制。对于$q$元LDPC分组码,采用FFT-QSPA解码算法以及停止规则为最大迭代次数$I_{max}=100$。对于$q$元LDPC卷积码,滑动窗口解码使用FFT-QSPA解码算法,停止规则为最大迭代次数$I_{max}=100$或者误符号率SER达到$10^{-6}$。
\chapter{总结}
本文在GF($q$)上,基于LDPC分组码以及基于原模图构造的有限长LDPC卷积码展开讨论,并比较了两者的性能。

为了公平地比较了两者的性能,本文在第二章通过类似的构造方式,即基于原模图复制和连接,构造了LDPC分组码和LDPC卷积码,并保证两者的解码延时相等。
本文在第三章简单描述了用于解码LDPC分组码的和积算法,并介绍了用于解码LDPC卷积码的滑动窗口算法。

通过实现以上算法进行解码实验,模拟结果表明LDPC卷积码所隐含的卷积特性使其性能超过LDPC分组码。
另外,本文还比较了,原模图复制数、解码窗口大小以及误比特率等参数之间的联系。
从中,我们可以得到多元LDPC卷积码性能比二元LDPC卷积码、多元LDPC分组码的性能更好。
在解码延时确定的条件下,原模图复制数、解码窗口大小等参数的组合导致不同的误比特率,故应根据信道及校验矩阵等具体情况确定最优组合。本文将为以后的应用提供参考数据。
\include{tikz}
\include{references}
% -*- coding: utf-8 -*-

%\makeschapterhead{致谢}
\chapter*{致谢}
本研究及学位论文是在陈鲁生导师的亲切关怀和悉心指导下完成的。他严肃的科学态度,严谨的治学精神,精益求精的工作作风,深深地感染和激励着我。从课题的选择到项目的最终完成,陈老师都始终给予我细心的指导和不懈的支持。这半年多来,陈老师不仅在学业上给我以精心指导,同时还在思想、生活上给我以无微不至的关怀,在此谨向陈老师致以诚挚的谢意和崇高的敬意。

在此,我还要感谢在一起愉快的度过本科四年生活各位同门,正是由于你们的帮助和支持,我才能克服一个一个的困难和疑惑,直至本文的顺利完成。

在论文即将完成之际,我的心情无法平静,从开始进入课题到论文的顺利完成,有多少可敬的师长、同学、朋友给了我无言的帮助,在这里请接受我诚挚的谢意!

最后我还要感谢培养我长大含辛茹苦的父母,谢谢你们!

\include{appendices}
\include{resume}
\end{document}
