% -*- coding: utf-8 -*-


\begin{zhaiyao}
本文介绍了GF(q)上LDPC分组码,并从LDPC分组码构建结构相似的卷积码。
接着分析了LDPC分组码的置信传播算法的性能,以及用于LDPC卷积码的滑动窗口版本置信传播算法的性能。
在不同参数的GF(q)上,本文考虑了两种情况,一种情况是LDPC分组码块长度与LDPC卷积码约束长度相等,另一种情况是解码延时相等,比较LDPC分组码与LDPC卷积码的解码性能。
其中模拟的结果表明q元LDPC卷积码的解码性能好与2元LDPC卷积码。
同时q元LDPC卷积码的解码性能好与q元LDPC分组码。
\end{zhaiyao}


\begin{guanjianci}
LDPC分组码;LDPC卷积码;滑动窗口;置信传播算法;解码延时
\end{guanjianci}



\begin{abstract}
In this paper, we introduce LDPC block codes (LDPC-BC), and LDPC convolutional codes (LDPC-CC) which are derived form the former. Then we analyse belief propagation (BP) decoder for LDPC-BC, and a sliding window decoder for LDPC-CC. Base on different GF(q), we compare the decoding performance between q-ary LDPC-BC and LDPC-CC in two aspects, one is the block length of q-ary LDPC-BC equals to constraint length of LDPC-CC, another is when two decoding latencies equal. Simulation shows that q-ary LDPC-CC outperform binary LDPC-CC and q-ary LDPC-BC.
\end{abstract}


\begin{keywords}
LDPC block codes; LDPC convolutional codes; sliding window decoder; belief propagation; decoding latencie.
\end{keywords} 