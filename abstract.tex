% -*- coding: utf-8 -*-


\begin{zhaiyao}
本文介绍了GF(q)上的LDPC分组码,并从LDPC分组码构建结构相似的LDPC卷积码。
接着分析了LDPC分组码的和积算法,以及用于LDPC卷积码的滑动窗口和积算法。
在不同参数的GF(q)上,本文主要考虑了两种情况:一种是LDPC分组码解码延时与LDPC卷积码的解码延时相等的情况;另一种是LDPC分组码码长与LDPC卷积码的限制长度相等的情况。
基于以上条件,本文比较了多元LDPC分组码与二元、多元LDPC卷积码的解码性能,
其中模拟的结果表明q元LDPC卷积码的解码性能好于2元LDPC卷积码,以及q元LDPC卷积码的解码性能好于q元LDPC分组码。
最后,本文还在解码延时相等的条件下,比较了q元LDPC卷积码与q元LDPC分组码的解码计算复杂度。
模拟结果表明,解码性能与计算复杂度两种条件的平衡中,LDPC卷积码较LDPC分组码表现更优。
\newline

\end{zhaiyao}


\begin{guanjianci}
LDPC分组码;LDPC卷积码;滑动窗口;和积算法;解码延时
\end{guanjianci}



\begin{abstract}
In this paper, we introduce LDPC block codes (LDPC-BC), and LDPC convolutional codes (LDPC-CC) which are derived form the former. Then we analyse sum product algorithm(SPA) for LDPC-BC, and a sliding window decoder(WD) for LDPC-CC. Base on different GF(q), we compare the decoding performance between q-ary LDPC-BC and LDPC-CC in two regimes: one when the constraint length of q-ary SC-LDPC codes is equal to the block length of q-ary LDPC-BCs and the other when the two decoding latencies are equal. Simulation shows that q-ary LDPC-CC outperform binary LDPC-CC and q-ary LDPC-BC.
We also compared computational complexity of q-ary LDPC-CC and q-ary LDPC-BC under equal decoding latency assumptions.
Simulation shows that 4-ary LDPC-CC outperform other LDPC-CCs and LDPC-BCs in trade-off between computational complexity and decoding latency.
\newline

\end{abstract}


\begin{keywords}
LDPC block codes; LDPC convolutional codes; Sliding window decoder; Belief propagation; Decoding latencie
\end{keywords} 