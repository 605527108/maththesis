\documentclass{beamer}
\usepackage{ctex}
\usepackage{tikz}
\usepackage{pgfplots}
\usepackage{beamerthemesplit}
\title{GF(q)上的LDPC分组码和LDPC卷积码的比较}
\author{孙卓豪}
\institute{南开大学 \& 电子信息科学与技术}
\date{\today}

\begin{document}
\usetikzlibrary{shapes,snakes}
\usetikzlibrary{arrows, decorations.markings}
%%%%%%%%%%%%%%%%%%%%%%
\begin{frame}
	\titlepage
\end{frame}
%%%%%%%%%%%%%%%%%%%%%%

%%%%%%%%%%%%%%%%%%%%%%
\begin{frame}
	\tableofcontents
\end{frame}
%%%%%%%%%%%%%%%%%%%%%%

\section{LDPC卷积码与LDPC分组码}
%%%%%%%%%%%%%%%%%%%%%%
\begin{frame}[shrink]
\frametitle{LDPC卷积码与LDPC分组码}
\begin{equation*}
    \mathbf{H}_{BC} = \left[
          \begin{array}{cc}
            \mathbf{H}_0 & \mathbf{H}_1\\
            \mathbf{H}_1 & \mathbf{H}_0
          \end{array} \right]_{2(c-b)M \times 2cM}
\end{equation*}
	\begin{equation*}
    \mathbf{H}_{CC} = \left[
          \begin{array}{ccccc}
            \mathbf{H}_0 & & & & \\
            \mathbf{H}_1 & \mathbf{H}_0 & & & \\
             & \mathbf{H}_1 & \mathbf{H}_0 & & \\
             & & \mathbf{H}_1 & \mathbf{H}_0 & \\
             & & & \mathbf{H}_1 & \ddots \\
             & & & & \ddots
          \end{array} \right]
\end{equation*}
\end{frame}
%%%%%%%%%%%%%%%%%%%%%%

\section{本文的主要工作}
%%%%%%%%%%%%%%%%%%%%%%
\begin{frame}[shrink]
\frametitle{本文的主要工作}
    \begin{itemize}
        \item 通过叠加原模图的方式,构建结构类似的LDPC卷积码与LDPC分组码;
        \item 描述解码LDPC分组码的和积算法,还描述了解码LDPC卷积码的滑动窗口算法,并计算两种算法的解码延时;
        \item 在LDPC分组码码长与LDPC卷积码限制长度相等的条件下,比较LDPC卷积码与LDPC分组码的性能;
        \item 在LDPC分组码与LDPC卷积码的解码延时相等的条件下,比较LDPC卷积码与LDPC分组码的性能;
        \item 比较LDPC分组码与LDPC卷积码的解码复杂度。
    \end{itemize}
\end{frame}
%%%%%%%%%%%%%%%%%%%%%%

\section{模拟仿真结果}
\subsection{LDPC分组码码长与LDPC卷积码限制长度相等}
%%%%%%%%%%%%%%%%%%%%%%
\begin{frame}[shrink]
	\frametitle{模拟仿真结果}
		\begin{block}{LDPC分组码码长与LDPC卷积码限制长度相等}
		\begin{columns}
		\begin{column}{0.4\textwidth}
		右图展示的是$q$元LDPC分组码及$q$元LDPC卷积码解码时误比特率BER达到$10^{-4}$所需翻转概率$\epsilon$与原模图复制数$M$之间的关系。
$q$元LDPC分组码及$q$元LDPC卷积码都由码率为$R=1/2$的(2,4)正则原模图构造。
$q$元LDPC卷积码的滑窗解码器的窗口大小为$W=12$。
		\end{column}
		\begin{column}{0.6\textwidth}
		\begin{center}
\pgfplotsset{compat=1.13}
\begin{tikzpicture} 
\begin{axis}[
    xlabel={原模图复制数$M$},
    ylabel={所需翻转概率$\epsilon$},
    grid=major,
    legend style={
        font=\tiny,
        cells={anchor=east},
        legend pos=north west,
    },
]
\addplot[color=red,mark=square] table {data_d15.dat};
\addplot[color=red,mark=triangle] table {data_d16.dat};
\addplot[color=red,mark=o] table {data_d17.dat};
\addplot[color=blue,mark=square] table {data_d18.dat};
\addplot[color=blue,mark=triangle] table {data_d19.dat};
\addplot[color=blue,mark=o] table {data_d20.dat};
\legend{2-ary LDPC-BC,
        4-ary LDPC-BC,
        8-ary LDPC-BC,
        2-ary LDPC-CC,
        4-ary LDPC-CC,
        8-ary LDPC-CC}
\end{axis}
\end{tikzpicture}
\end{center}
\end{column}
\end{columns}
		\end{block}
\end{frame}
%%%%%%%%%%%%%%%%%%%%%%
%%%%%%%%%%%%%%%%%%%%%%
\begin{frame}[shrink]
	\frametitle{模拟仿真结果}
		\begin{block}{LDPC分组码码长与LDPC卷积码限制长度相等}
		\begin{columns}
		\begin{column}{0.4\textwidth}
		右图描述的是$q$元LDPC分组码及$q$元LDPC卷积码解码时误比特率BER达到$10^{-4}$所需翻转概率$\epsilon$与原模图复制数$M$之间的关系。
$q$元LDPC分组码及$q$元LDPC卷积码都由$R=1/2$的(3,6)正则原模图构造。
		\end{column}
		\begin{column}{0.6\textwidth}
		\begin{center}
\pgfplotsset{compat=1.13}
\begin{tikzpicture} 
\begin{axis}[
    xlabel={原模图复制数$M$},
    ylabel={所需翻转概率$\epsilon$},
    grid=major,
    legend style={
        font=\tiny,
        cells={anchor=east},
        legend pos=north west,
    },
]
\addplot[color=red,mark=square] table {data_d21.dat};
\addplot[color=red,mark=triangle] table {data_d22.dat};
\addplot[color=red,mark=o] table {data_d23.dat};
\addplot[color=blue,mark=square] table {data_d24.dat};
\addplot[color=blue,mark=triangle] table {data_d25.dat};
\addplot[color=blue,mark=o] table {data_d26.dat};
\legend{2-ary LDPC-BC,
        4-ary LDPC-BC,
        8-ary LDPC-BC,
        2-ary LDPC-CC,
        4-ary LDPC-CC,
        8-ary LDPC-CC}
\end{axis}
\end{tikzpicture}
\end{center}
		\end{column}
		\end{columns}
		\end{block}
\end{frame}
%%%%%%%%%%%%%%%%%%%%%%

\subsection{LDPC分组码与LDPC卷积码的解码延时相等}
%%%%%%%%%%%%%%%%%%%%%%
\begin{frame}[shrink]
	\frametitle{模拟仿真结果}
		\begin{block}{LDPC分组码与LDPC卷积码的解码延时相等}
		\begin{columns}
		\begin{column}{0.4\textwidth}
右图中8元LDPC分组码及8元LDPC卷积码都由码率为$R=1/2$的(3,6)正则原模图构造。
		\end{column}
		\begin{column}{0.6\textwidth}
		\begin{center}
		\pgfplotsset{compat=1.13}
\begin{tikzpicture}
\begin{semilogyaxis}[
    xlabel={反转概率$\epsilon$},
    ylabel={BER},
    grid=major,
    legend style={
        font=\tiny,
        cells={anchor=west},
        legend pos=south east,
    },
    ymin=0.00001,
    ymax=1,
]
\addplot[color=red,mark=square] table {data_d1.dat};
\addplot[color=red,mark=triangle] table {data_d2.dat};
\addplot[color=blue,mark=o] table {data_d3.dat};
\addplot[color=blue,mark=square] table {data_d4.dat};
\addplot[color=blue,mark=triangle] table {data_d5.dat};
\addplot[color=blue,mark=diamond] table {data_d6.dat};
\legend{LDPC-BC $M=192$\\
        LDPC-BC $M=384$\\
        LDPC-CC $M = 32,W =12$\\
        LDPC-CC $M = 64,W =6$\\
        LDPC-CC $M = 64,W =12$\\
        LDPC-CC $M = 128,W =6$\\}
\end{semilogyaxis}
\end{tikzpicture}
\end{center}
		\end{column}
		\end{columns}
		\end{block}
\end{frame}
%%%%%%%%%%%%%%%%%%%%%%
%%%%%%%%%%%%%%%%%%%%%%
\begin{frame}[shrink]
	\frametitle{模拟仿真结果}
		\begin{block}{LDPC分组码与LDPC卷积码的解码延时相等}
		\begin{columns}
		\begin{column}{0.4\textwidth}
		右图展示在GF($8$)上比较要达到相同的误比特率所需的翻转概率与解码延时之间的关系。
其中8元LDPC分组码及8元LDPC卷积码都由码率为$R=1/2$的(3,6)正则原模图构造。
		\end{column}
		\begin{column}{0.6\textwidth}
		\begin{center}
		\begin{tikzpicture} 
\begin{axis}[
    xlabel={解码延时(bits)},
    ylabel={所需翻转概率$\epsilon$},
    grid=major,
    legend style={
        font=\tiny,
        cells={anchor=west},
        legend pos=south east,
    },
    xmin=0,
]
\addplot table {data_d7.dat};
\addplot table {data_d8.dat};
\addplot table {data_d9.dat};
\addplot table {data_d10.dat};
\legend{LDPC-BC,$M = 32,64,128,192,288,384,576$\\
        LDPC-CC,$M = 32,W =2,3,\dots,12$\\
        LDPC-CC,$M = 64,W =2,3,\dots,12$\\
        LDPC-CC,$M = 128,W =2,3,\dots,8$\\}
\end{axis}
\end{tikzpicture}
\end{center}
		\end{column}
		\end{columns}
		\end{block}
\end{frame}
%%%%%%%%%%%%%%%%%%%%%%
%%%%%%%%%%%%%%%%%%%%%%
\begin{frame}[shrink]
	\frametitle{模拟仿真结果}
		\begin{block}{LDPC分组码与LDPC卷积码的解码延时相等}
		\begin{center}
		\begin{tabular}{|c|c|c|c|c|c|c|}
 \hline
解码 & \multicolumn{3}{|c|}{LDPC-BC} & \multicolumn{3}{|c|}{LDPC-CC} \\ \cline{2-7}
延时 & GF(2) & GF(4) & GF(8) & GF(2) & GF(4) & GF(8) \\ \hline
2304 bits & 0.14 & 0.15 & 0.15 & 0.12 & 0.16 & 0.18\\ \hline
4608 bits & 0.16 & 0.17 & 0.16 & 0.18 & 0.19 & 2.0\\ \hline
6912 bits & 0.17 & 0.18 & 0.17 & 0.19 & 0.21 & 0.21\\ \hline
9216 bits & 0.17 & 0.19 & 0.19 & 0.21 & 0.22 & 0.22\\ \hline
\end{tabular}
\end{center}
上表展示了,(3,6)正则q元LDPC分组码与(3,6)正则q元LDPC卷积码在不同的GF($q$)以及不同的解码延时的条件下,误比特率BER达到$10^{-4}$所需翻转概率$\epsilon$。
		\end{block}
\end{frame}
%%%%%%%%%%%%%%%%%%%%%%


\subsection{LDPC分组码与LDPC卷积码的解码复杂度比较}
%%%%%%%%%%%%%%%%%%%%%%
\begin{frame}[shrink]
	\frametitle{模拟仿真结果}
		\begin{block}{LDPC分组码与LDPC卷积码的解码复杂度比较}
		\begin{columns}
		\begin{column}{0.4\textwidth}
		右图展示了,(3,6)正则q元LDPC卷积码与(3,6)正则q元LDPC分组码每解码一比特的计算复杂度与GF($q$)的关系。
		\end{column}
		\begin{column}{0.6\textwidth}
		\begin{center}
\begin{tikzpicture} 
\begin{axis}[
    xlabel={GF($q$)},
    ylabel={解码每一比特的计算复杂度},
    grid=major,
    legend style={
        font=\tiny,
        cells={anchor=west},
        legend pos=north west,
    },
    xmin=0,
    ymin=0,
]
\addplot table {data_d27.dat};
\addplot table {data_d28.dat};
\addplot table {data_d29.dat};
\addplot table {data_d30.dat};
\legend{4608 LDPC-BC\\
        6912 LDPC-BC\\
        4608 LDPC-CC\\
        6912 LDPC-CC\\}
\end{axis}
\end{tikzpicture}
\end{center}
		\end{column}
		\end{columns}
		\end{block}
\end{frame}
%%%%%%%%%%%%%%%%%%%%%%
%%%%%%%%%%%%%%%%%%%%%%
\begin{frame}[shrink]
	\frametitle{模拟仿真结果}
		\begin{block}{LDPC分组码与LDPC卷积码的解码复杂度比较}
		\begin{center}
		\begin{tabular}{|c|c|c|c|c|c|c|}
 \hline
解码 & \multicolumn{3}{|c|}{$I_{BC}$} & \multicolumn{3}{|c|}{$I_{CC}(W=6)$} \\ \cline{2-7}
延时 & GF(2) & GF(4) & GF(8) & GF(2) & GF(4) & GF(8) \\ \hline
4608 bits & 13.8 & 12.3 & 11.1 & 3.3 & 3.2 & 3.0\\ \hline
6912 bits & 15.6 & 14.1 & 12.6 & 3.9 & 3.7 & 3.4\\ \hline
\end{tabular}
\end{center}
基于相等的解码延时的条件,比较q元LDPC卷积码与q元LDPC分组码的计算复杂度。
上表展示了,在解码延时分别为4608,6912比特时,(3,6)正则q元LDPC卷积码与(3,6)正则q元LDPC分组码解码误比特率达到$10^{-4}$所需的平均迭代次数$I_{CC}$和$I_{BC}$。
		\end{block}
\end{frame}
%%%%%%%%%%%%%%%%%%%%%%
\end{document}