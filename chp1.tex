\chapter{背景及介绍}
Gallager在1962年提出低密度校验分组码\parencite{Gallager1963Low},同时他还提出了一种信息遍历解码算法。由于当时计算机计算能力不高,LDPC码并未得到重视,直到九十年代人们才重新认识到LDPC码的优点。
在2001年Richardson等人的模拟试验中\parencite{Richardson2001Design},用信息遍历算法解码LDPC分组码的性能超过了turbo码,并且接近香农极限。
而Davey和MacKay考虑了定义在GF($q$)的LDPC分组码\parencite{Davey1998Low},同时把Gallager的BP解码算法推广到q元域,称作q元和积算法,并得到很好的性能。
1999年,Davey将快速傅立叶变换引入和积算法中,使得和积算法的复杂度有所降低\parencite{Davey1999Error}。
另外,2004年Wymeersch在和积算法引入对数似然比的概念,避免了实数乘法运算\parencite{Wymeersch2004Log}。

与LDPC分组码相对应的卷积码,称为LDPC卷积码,由Felström和Zigangirov\parencite{Felstrom1999Time}于1999年提出。
LDPC卷积码与LDPC分组码类似,能通过稀疏校验矩阵定义,所以LDPC卷积码解码也可以用信息遍历解码算法。
LDPC卷积码的特点是它的BP解码阈值能达到它所基于的LDPC码的最大后验概率(MAP)解码阈值\parencite{Kudekar2010Threshold},简称阈值饱和现象。
Pusane等人在\parencite{Pusane2008Implementation}专门为LDPC卷积码设计了并行的流水线解码算法,
但是为了达到接近香农极限的性能,该算法必须经过多次迭代,将占用大量内存,同时有很高的译码延时。
为了解决LDPC卷积码顺序解码时以上缺点,Papaleo等人考虑了BP算法窗口解码版本\parencite{Papaleo2010Windowed},并研究了算法中解码延时和解码性能之间的取舍。

在构建LDPC卷积码方面,Felström和Zigangirov做出了开创性工作\parencite{Felstrom1999Time}。
随后基于原模图parencite{Thorpe2003Low}构造LDPC卷积码的方法被提出。结合原模图与progressive edge growth算法\parencite{Hu2005Regular},LDPC卷积码能很简单地构造出来。

由于LDPC卷积码与LDPC分组码具有相似的结构,Costello在\parencite{Costello2006A}与\parencite{Costello2007A}中比较了二元LDPC分组码与二元LDPC卷积码的性能。
Hassan等人基于解码延时\parencite{Ul2012Comparison}和Hehn等人基于结构延时\parencite{Hehn2009LDPC}研究LDPC分组码与LDPC卷积码的特点。
本文基于以上的研究进行q元域的推广,同时比较其他相关参数对LDPC卷积码与LDPC分组码解码性能的影响。





















本文的结构如下:第一章主要介绍LDPC卷积码与LDPC分组码的历史及背景。
第二章主要描述LDPC卷积码与LDPC分组码的构建过程,同时指出公平比较两者性能所基于的构建规则。
第三章主要描述LDPC卷积码与LDPC分组码的解码算法,并且计算了两种算法的解码延时。
第四章通过调整几类参数,比较LDPC卷积码与LDPC分组码的性能。
第五章是结论及总结以上内容。