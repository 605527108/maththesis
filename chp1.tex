\chapter{背景及介绍}
\section{背景}
低密度校验分组码(LDPC-BC),当其结合使用复杂度低的置信传播解码算法时,是一类接近香农极限的编码。
另外,它还有另一个优点,即其解码复杂度随着分组码码长增加而线性增加。
Gallager在1962年提出低密度校验分组码\parencite{1057683}。
在同一篇文章中,他还提出了一种信息遍历解码算法。
由于当时计算机计算能力不高,LDPC码并未得到重视,直到九十年代人们才重新认识到LDPC码的优点。
在2001年Richardson等人的模拟试验中\parencite{910578},用信息遍历算法解码LDPC分组码的性能超过了turbo码,并且接近香农极限。
而Davey和MacKay考虑了定义在有限域GF($q$)的LDPC分组码\parencite{681360},同时把Gallager的置信传播解码算法推广到q元域,称作q元和积算法(QSPA)。
该文章中的实验证明q元LDPC分组码具有很好的性能。
2003年,Barnault等人将快速傅立叶变换引入和积算法中,将和积算法推广为FFT-QSPA,使得和积算法的复杂度有所降低\parencite{1216697}。
另外,2004年Wymeersch在和积算法引入对数似然比的概念,避免了实数乘法运算\parencite{1312606}。
之后,拓展最小和算法(EMS)被提出,它能进一步降低解码的复杂度。
由于码长中等和码长较短的LDPC分组码有极好的解码性能,q元LDPC分组码最近获得学术界极大关注。

与LDPC分组码相对应的卷积码,称为LDPC卷积码,由Felström和Zigangirov\parencite{782171}于1999年提出。
LDPC卷积码与LDPC分组码类似,能通过稀疏校验矩阵定义,所以LDPC卷积码解码也可以用信息遍历解码算法。
LDPC卷积码的特点是,如果该LDPC卷积码是从某一类正则或非正则LDPC分组码衍生构造的话,
LDPC卷积码的置信传播解码算法阈值能渐近达到,它所基于的LDPC分组码的最大后验概率(MAP)解码阈值\parencite{5695130}。
随后,某一类从正则LDPC分组码衍生构造出来的LDPC卷积码被证明存在阈值饱和现象,即LDPC卷积码如果在无记忆对称二元信道中使用置信传播解码算法的话,能够达到它所基于的LDPC分组码的MAP阈值,也就是说正则LDPC卷积码能够通过增加校验矩阵的密度达到信道容量。

由于LDPC卷积码与LDPC分组码具有相似的结构,Costello在\parencite{4357569}中比较了二元LDPC分组码与二元LDPC卷积码的性能。
在这篇文章中,Costello使用了一种并行高速流水线解码算法,这种算法的具体实现方式在Pusane等人的\parencite{4568447}中有深入的介绍。
但是为了达到接近香农极限的性能,该算法必须经过多次迭代,而解码延时和内存占用量与算法迭代次数正相关,所以,要达到香农极限,其内存占用量与译码延时大大影响了解码效率。
为了解决LDPC卷积码使用流水线解码算法时的缺点,Iyengar等人提出了置信传播解码算法的滑动窗口版本\parencite{6086762},并研究了算法中解码延时和解码性能之间的取舍。
在\parencite{6325232}中,Hassan等人将改进后的滑动窗口解码算法应用于迭代解码阈值分析。
在\parencite{5089507}中,Hehn等人基于结构延时,对LDPC分组码与LDPC卷积码进行了比较,其中LDPC卷积码使用滑动窗口解码算法进行解码。

在构建LDPC卷积码方面,Felström和Zigangirov做出了开创性工作\parencite{782171}。
并且,在这篇文章中,Felström和Zigangiro证明了在二元擦除信道中,q元LDPC卷积码也具有二元LDPC卷积码阈值饱和现象。
随后基于原模图\parencite{Thorpe2003Low}构造LDPC卷积码的方法被提出。
结合原模图与边逐步增长方法(PEG)\parencite{1377521},LDPC卷积码能很方便地构造出来。
最近,\parencite{6874959}中分析了由原模图构造的q元LDPC卷积码在滑动窗口解码算法中的解码阈值表现。
因此这篇文章也为设计与实现适用于滑动窗口解码算法的q元LDPC卷积码提供了理论指导。

相对于Hehn等人的\parencite{5089507}所考虑的二元LDPC卷积码的解码性能不同,
本文主要考虑有限长度的q元LDPC卷积码以及q元LDPC分组码,模拟仿真所基于的信道是二元对称信道。
由于流水线解码算法会造成较大的解码延时,本文在解码q元LDPC卷积码时主要采用滑动窗口解码算法。
另外,为了减小计算的复杂度,本文还引入了一个基于置信传播软判决的迭代停止规则。
在比较q元LDPC卷积码与q元LDPC分组码的性能时,本文主要考虑两种情况:一种情况是q元LDPC卷积码的限制长度等于q元LDPC分组码的码长;
另一种情况是q元LDPC卷积码的解码延时与q元LDPC分组码的解码延时相等。
本文在q元域上的比较方法与\parencite{5089507}中二元域的比较方法类似,但是本文也有自己的重点。
比如,对于由(2,4)正则LDPC码构造的q元LDPC分组码与q元LDPC卷积码之间的比较,我们将码长参数限制在相对小的区间。
本文还考察了,当解码延时确定时,原模图复制数,解码窗口大小与q元LDPC卷积码解码性能之间的关系。
最后,本文还在解码性能或者解码延时确定的条件下,比较q元LDPC分组码与q元LDPC卷积码之间解码算法的计算复杂度。
\section{本文结构}
本文的结构如下:第一章主要介绍LDPC卷积码与LDPC分组码的历史及背景。
第二章主要描述LDPC卷积码与LDPC分组码的构建过程,同时指出公平比较两者性能所基于的构建规则。
第三章主要描述LDPC卷积码与LDPC分组码的解码算法,并且计算了两种算法的解码延时。
在第四章,通过调整几类参数,我们比较了LDPC卷积码与LDPC分组码的性能。
第五章是结论及总结以上内容。