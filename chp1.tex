\chapter{背景及介绍}
Gallager在1962年提出低密度校验分组码\cite{Gallager1963Low},同时他还提出了一种信息遍历解码算法。
在Richardson等人模拟试验中\cite{keylist},用信息遍历算法解码LDPC分组码的性能超过了turbo码,并且接近香农极限。
Davey和MacKay考虑了定义在GF(q)的LDPC分组码\cite{Davey1998Low}并且把Gallager的BP解码算法推广到q元,称作和积算法,并得到很好的性能。

由Felström和Zigangirov\cite{Felstrom1999Time}提出来的LDPC卷积码可以通过几个正则LDPC码叠加连接而成。
LDPC卷积码的特点是它的BP解码阈值能达到它所基于的LDPC码的最大后验概率(MAP)解码阈值\cite{Kudekar2010Threshold}。
但是为了达到接近香农极限的性能,BP算法必须经过多次迭代,将占用大量内存同时有很高的译码延时。
为了解决LDPC卷积码顺序解码时以上缺点,Papaleo等人考虑了BP算法窗口解码版本\cite{Papaleo2010Windowed},并研究了算法中解码延时和解码性能之间的取舍。

由于LDPC卷积码与LDPC分组码具有相似的结构,并且在解码延时和解码性能互有取舍,Hassan等人基于解码延时\cite{Ul2012Comparison}和Hehn等人基于结构延时\cite{Hehn2009LDPC}研究其二者的性能比较。
本文基于以上的研究进行q元域的推广,同时比较其他相关参数对LDPC卷积码与LDPC分组码解码性能的影响。
本文的结构如下:第一章主要是LDPC卷积码与LDPC分组码背景及介绍。
第二章主要是描述LDPC卷积码与LDPC分组码的构建过程,要在特殊的构建原则上两者才能比较。
第三章主要描述LDPC卷积码与LDPC分组码的解码算法,并且计算了两种算法的解码延时。
第四章通过固定几个参数比较另外的参数,比较LDPC卷积码与LDPC分组码的性能。
第五章是结论及总结。